%---------------DOCUMENT SETTINGS------------------------------------
%\documentclass[headheight=30pt]{scrartcl}
\documentclass{article}

\usepackage[utf8]{inputenc} % utf8x durch utf8 ersetzt wegen biblatex
\usepackage[T1]{fontenc}
\usepackage{amsmath,amssymb,amstext,amsfonts}
\usepackage[ngerman]{babel}
\usepackage{csquotes}
\usepackage{pdfpages} %zum importieren des Deckblattes
\usepackage{geometry}
\usepackage[headsepline]{scrlayer-scrpage}
\usepackage{lastpage}
\usepackage[style=numeric]{biblatex} %Literaturverwaltung
\usepackage{tabularx} %für die Legende im Gruundlagen Oszi Bild
\usepackage{tabulary}
\usepackage{float}
\usepackage{textcomp}
\usepackage{gensymb}
\usepackage{physics}
\usepackage{stmaryrd}
\usepackage{graphicx}
\usepackage[export]{adjustbox}
\usepackage{a4wide}
\usepackage{siunitx}
\usepackage{hyperref}
\usepackage{multicol}
\usepackage{makecell}
\usepackage{enumitem}
\usepackage{subcaption}


\geometry{      %holt mehr aus einer A4 Seite raus
    a4paper,
    total={170mm,257mm},
    left=20mm,
    top=25mm,
   }
\graphicspath{ {./graphics/} }  %pfad für bilder
\hypersetup{colorlinks=false}
\addbibresource{literatur.bib} %Literatur-Resourcen\s
%\setlength{\headheight}{0.0pt} % macht zwar headheight warnings aber dafür nutz latex die seitengröße besser aus.
\pagestyle{scrheadings}
\newcommand{\subf}[2]{
    {
        \begin{tabular}[c]{@{}c@{}}
            {\setlength{\extrarowheight}{100pt} #1 }\\#2
        \end{tabular}
    }
}
\newcommand{\allc}{\multicolumn{1}{c|}{-}}
\newcommand{\monofig}[4]{
    
    \begin{figure}[H]
    \centering
    \includegraphics[#1]{#2}
    \caption{
        #3
    }
    \label{#4}
    \end{figure}
    
}
\newcommand{\polyfig}[6]{
    \begin{figure}[H]
        \centering
        \begin{minipage}[b]{0.45\textwidth}
            \centering
            \includegraphics[width=\textwidth]{#1}
            \caption{#2}
            \label{#3}
        \end{minipage}
        \begin{minipage}[b]{0.45\textwidth}
            \centering
            \includegraphics[width=\textwidth]{#4}
            \caption{#5}
            \label{#6}
        \end{minipage}
    \end{figure}
}
\newcommand{\polyfigacc}[8]{
    \begin{figure}[H]
        \centering
        \begin{minipage}[b]{#1}
            \centering
            \includegraphics[width=\textwidth]{#2}
            \caption{#3}
            \label{#4}
        \end{minipage}
        \begin{minipage}[b]{#5}
            \centering
            \includegraphics[width=\textwidth]{#6}
            \caption{#7}
            \label{#8}
        \end{minipage}
    \end{figure}
}
\newcommand{\trifig}[9]{
    \begin{figure}[H]
        \centering
        \begin{minipage}[b]{0.32\textwidth}
            \centering
            \includegraphics[width=\textwidth]{#1}
            \caption{#2}
            \label{#3}
        \end{minipage}
        \begin{minipage}[b]{0.32\textwidth}
            \centering
            \includegraphics[width=\textwidth]{#4}
            \caption{#5}
            \label{#6}
        \end{minipage}
        \begin{minipage}[b]{0.32\textwidth}
            \centering
            \includegraphics[width=\textwidth]{#7}
            \caption{#8}
            \label{#9}
        \end{minipage}
        \caption{#10}
        \label{#11}
    \end{figure}
}
\newcommand{\trifigcom}[8]{
    \begin{figure}[H]
        \centering
        \begin{subfigure}[b]{0.32\textwidth}
            \centering
            \includegraphics[width=\textwidth]{#1}
            \caption{}
            \label{#2}
        \end{subfigure}
        \begin{subfigure}[b]{0.32\textwidth}
            \centering
            \includegraphics[width=\textwidth]{#3}
            \caption{}
            \label{#4}
        \end{subfigure}
        \begin{subfigure}[b]{0.32\textwidth}
            \centering
            \includegraphics[width=\textwidth]{#5}
            \caption{}
            \label{#6}
        \end{subfigure}
        \caption{#7}
        \label{#8}
    \end{figure}
}
\newcommand{\duofigcom}[6]{
    \begin{figure}[H]
        \centering
        \begin{subfigure}{0.49\textwidth}
            \includegraphics[width=\textwidth]{#1}
            \caption{}
            \label{#2}
        \end{subfigure}
        \begin{subfigure}{0.49\textwidth}
            \includegraphics[width=\textwidth]{#3}
            \caption{}
            \label{#4}
        \end{subfigure}
        \caption{
            #5
            }
        \label{#6}
    \end{figure}
}
\date{\today{}, Location}
\author{Aleksey Sokolov, Max Jost}
\title{Interferenz und Polarisation}

%---------------HEADER TEXT------------------------------------
\clearpairofpagestyles
\ihead{\today{} \\ }
\chead{Max Jost / Aleksey Sokolov\\ Interferenz und Polarisation }
\ohead{FPTP 2 \\ }
\cfoot{\pagemark \, / \, \pageref{LastPage}}

%---------------DOCUMENT TEXT------------------------------------
\begin{document}
\includepdf[]{Deckblatt.pdf} %Insert title page NaWi-Graz
\tableofcontents
\newpage

%%***************ANMERKUNGEN*******************
\section*{Anmerkung}
\label{sec:anmerkung}
Durch die aktuelle globale COVID-19 Pandemie ist es uns nicht möglich diese Laborübung in einem Labor der Universität durchzuführen.
Aufgrund dessen machen wir in diesem Semester eine Home-Lab-Übung.
Durch die Umstände erfolgte der Versuchsaufbau mit leichteren Mitteln. Dennoch wurde das Ziel der Übung erfüllt.
\section{Aufgabenstellung}
\label{sec:Aufgabenstellung}
\textit{Die Aufgabenstellung wurde aus \cite{LAB} entnommen.}
\begin{enumerate}
    \item Young'scher Doppelspalt, Beugungsgitter
    \begin{enumerate}
        \item Aufnehmen und vermessen der Beugungsmuster von vier Doppelspalten mit (bekannten) unterschiedlichen Spaltbreiten und Spaltabständen. Berechnen Sie aus den Messwerten die Wellenlänge des Lasers.
        \item Erklären der Details der beobachteten Beugungsmuster durch Vergleich mit den berechneten Mustern.
        \item Bestimmung des Beugungsmuster eines Liniengitters und Vergleich mit berechneten Werten. Berechnung der Gitterkonstante aus Messwerten.
    \end{enumerate}
    \item Wellenfront-Analyse. (Nicht möglich weil Shearing Interferometer in Reparatur)
    \item Polarisation
    \begin{enumerate}
        \item Verifikation des Gesetzes von Malus.
        \item Untersuchung des Einflusses des Durchlasswinkels eines dritten Polarisators zwischen zwei gekreuzten Polarisatoren.
    \end{enumerate}
    \item Michelson Interferometer
    \begin{enumerate}
        \item Justierung eines Interferometers und Erzeugung eines konzentrisches Interferenzmuster. Bestimmung der Wellenlänge des Lasers durch Weglängenänderung. Wiederholung der Messung für ein paralleles Interferenzmuster.
        \item Untersuchung des absoluten Weglängenunterschieds in den beiden Interferometerarmen, sowie Auflösung und Stabilität des Interferometers.
        \item Untersuchung der Rolle der Polarisation auf die Interferenzfähigkeit des Laserlichts
    \end{enumerate}
\end{enumerate}
\newpage
%********VORAUSSETZUNGEN & GRUNDLAGEN*********
\chapter{Fundumentals}
\label{sec:Fundumentals}
\section{STM-Imaging}
The Scanning Tunneling Microscope was introduced in 1981 by Gerd Binning and Heinrich Roherer. 
With this measuring technique it is possible to resolve a conductive surface with a precission beyond that of conventional light based Microscopes.
In contrast to other electron based microscopy like Scanning Electron Microscopes (SEM) it uses the quantum mechanical phenomenon of tunneling.
In classical mechanics, objects cannot overcome a potential if their energy $E < V_0$, as observed in gravitational interactions.
This phenomenon is observed for quantum mechanical particles like electrons, which can surpass a potential barrier despite the initial expectation that they should not be able to.
The STM uses this effect by precisely positioning a sharp conductive tip close to the surface and applying a bias voltage.
Most STM are operated in Ultra-High-Vacuum (UHV), where the distance between the tip and the surface represents the tunneling barrier.
By varying the bias voltage, the tunneling probability can be changed, thereby affecting the tunneling current.
If the bias voltage, also referred to as the potential difference, is kept constant, the tunneling current is primarily dependent on the distance between tip and surface.
The tip is moved in the x-,y-plane where a grid is established. 
There are two modes of operation, the constant-height and the constant-current mode.
The latter is especially useful for irregular surfaces, because the tip is moved up and down to keep the tunneling current constant.
The movement signal of the piezos is then converted into height.
In constant-height mode the position of the tip stayes fixed and the tunneling current $I_t$ is measured and converted into height information. 

\newpage
%************VERSUCHSANORDNUNG*************
\section{Versuchsanordnung}
\label{sec:versuchsandordnung}

\subsection{Young'scher Doppelspalt, Beugungsgitter}
\monofig{width=0.8\textwidth}{double_slit_aufb_label.jpg}{
    Aufbau der Young'schen Doppelspalt- und Beugungsgitter-Analyse.
    \textbf{L}: Laser ($\lambda$: 532 nm); 
    \textbf{S1/S2}: Spiegel; 
    \textbf{S/G}: Halterung für Doppelspalte und Gitter.
    }{fig:double_slit_aufb}

\subsection{Polaritäts-Analyse}
\monofig{width=0.8\textwidth}{pol_aufbau_beschriftet.jpg}{
    Aufbau für die Verifizierung des Gesetztes von Malus.
    \textbf{A}: Laser ($\lambda$: 532 nm),
    \textbf{B}: Spiegel (90° Umlenkung des Laserstrahls),
    \textbf{C}: 2x Linear Polarisatoren,
    \textbf{D}: Lux-Meter (Intensitätsbestimmung) }{fig:pol_aufbau}
\subsection{Michelson-Interferometer}
\monofig{width=0.8\textwidth}{michelson_aufbau_beschriftetet.jpg}{
    Aufbau Michelson-Interferometer.
    \textbf{A}: Linse ($f$ = 40 mm)
    \textbf{B}:Beamsplitter,
    \textbf{C}: verschiebbarer Interferometerspiegel,
    \textbf{D}: starrer Interferometerspiegel (Winkel verstellbar),
    \textbf{E}:Abbildungsschirm
}{fig:michelson_inter}
%************GERÄTELISTE***************
\section{Geräteliste}
\label{sec:geraeteliste}
\begin{table}[H]
	\centering
	\caption{
		Im Versuch verwendete Geräte und Utensilien.}
	\begin{tabularx}{1.0\textwidth}{| l | l | l | X |}
		\hline
		Gerät       & Hersteller   & Model  & technische Daten /\linebreak Unsicherheit       \\\hline
		\hline
		Laser & ThorLabs & CPS532-C2 & $\lambda$ = 532 nm; $P$ = 0.9 mW; $PER$ = 5 dB\\ \hline
		2 x Spiegel & ThorLabs & \allc & \allc \\ \hline
		4 x Doppelspalt & \allc & \allc & $b_1$ = 0.2 mm; $b_{2,3,4}$ = 0.1 mm; $a_{1,2}$ = 0.25 mm; $a_3$ = 0.5 mm; $a_4$ = 1 mm \\ \hline
		Beugungsgitter & \allc & \allc & \allc \\ \hline
		2 x Linearpolarisierungsfenster & \allc & \allc & \allc \\ \hline
		2 x Polarisatorhalterungen & ThorLabs & RSP05/M & \allc \\ \hline
		Linearpolarisierungsfolie & \allc & \allc & \allc \\ \hline
		Photodetektor & Sauter & SO 200K & \allc \\ \hline
		Linse & \allc & \allc & $f$ = 40 mm \\ \hline
		Michelson Interferometer & \allc & \allc & \allc \\ \hline
	\end{tabularx}
	\label{tab:Geraeteliste}
\end{table}
\newpage
%************VERSUCHSDURCHFÜHRUNG & MESSERGEBNISSE***********
\chapter{Methology}
\label{sec:methology}

\newpage
%************AUSWERTUNG****************
\section{Auswertung}
\label{sec:auswertung}
Die Auswertung wird mit der Skriptingsprache \verb|Python| \cite{PYTHON} erstellt.
In der gesamten Auswertung werden die Python Libraries, \verb|numpy| \cite{harris2020array}, \verb|pandas| \cite{reback2020pandas}, \verb|scipy| \cite{2020SciPy-NMeth} und \verb|matplotlib| \cite{Hunter:2007}, für statistische Rechenoperationen, das Einlesen und Verarbeiten von Daten sowie das Erstellen von Plots verwendet.
Genaue Funktionsnamen, die zur Berechnung von Messgrößen oder Fit-Parametern verwendet werden, sind in der weiteren Auswertung angegeben.
Des weiteren wird die Library \verb|uncertainties| \cite{UN} für jegliche Unsicherheitsfortpflanzungen (wenn nicht anders angegeben) benutzt, da diese auf dem Prinzip der Größtunsicherheitsmethode Gl. \ref{equ:groestunsicherheit} basiert.


\subsection{Young'scher Doppelspalt, Beugungsgitter}
Da die Aufnahmen aus Abb.\ref{fig:interf_muster} der Beugungsmuster eine Perspektivenverzerrung aufweisen wird diese vorerst mit der Software FIJI (Plugin "Transform -> Interactive Perspective") korrigiert.
Dabei orientiert man sich an einem Äquidistanten Gitter, welches über das Bild gelegt wird und korrigiert durch verzerren von 4 Punkten am Bild.
Die korrigierten Bilder mit vergleichsgitter sind in Abb.\ref{fig:interf_muster_corr_grid} zu sehen.
\begin{figure}[h]
    \centering
    \begin{subfigure}[t]{0.4\textwidth}
        \includegraphics[width=\textwidth]{double_slit_muster1_corr_grid.jpg}
        \caption{}
        \label{fig:double_slit_muster1_corr_grid}
    \end{subfigure}%
    \begin{subfigure}[t]{0.4\textwidth}
        \includegraphics[width=\textwidth]{double_slit_muster2_corr_grid.jpg}
        \caption{}
        \label{fig:double_slit_muster2_corr_grid}
    \end{subfigure}
    \begin{subfigure}[t]{0.4\textwidth}
        \includegraphics[width=\textwidth]{double_slit_muster3_corr_grid.jpg}
        \caption{}
        \label{fig:double_slit_muster3_corr_grid}
    \end{subfigure}%
    \begin{subfigure}[t]{0.4\textwidth}
        \includegraphics[width=\textwidth]{double_slit_muster4_corr_grid.jpg}
        \caption{}
        \label{fig:double_slit_muster4_corr_grid}
    \end{subfigure}
    \begin{subfigure}[t]{0.4\textwidth}
        \includegraphics[width=\textwidth]{gitter_muster_corr_grid.jpg}
        \caption{}
        \label{fig:gitter_muster_corr_grid}
    \end{subfigure}

    \caption{
        Perspektivenkorrigierte Aufnahmen der Interferenzmuster mit eingezeichnetem Vergleichsgitter.
        \ref{fig:double_slit_muster1_corr_grid}: Doppelspalt 1; 
        \ref{fig:double_slit_muster2_corr_grid}: Doppelspalt 2; 
        \ref{fig:double_slit_muster3_corr_grid}: Doppelspalt 3; 
        \ref{fig:double_slit_muster4_corr_grid}: Doppelspalt 4; 
        \ref{fig:gitter_muster_corr_grid}: Gitter.
    }
    \label{fig:interf_muster_corr_grid}
\end{figure}
An den korrigierten Aufnahmen werden dann mit der PIL(Python Image Library) die Pixel in ihren RGB CHannel Werten gefilter, so dass nur Pixel mit Wertigkeit größer als 160 überbleiben.
Dies ist nötig um ein gutes Interferenzsignal zu erhalten.
Danach wird aus jedem Bild nur der Bereich ausgeschnitten in dem das Interferenzmuster zu sehen ist.
Die Ausgeschnittenen Bereiche sind in Abb.\ref{fig:yng_cutouts} zu sehen.
\monofig{width=0.7\textwidth}{yng_cutouts.pdf}{
    Ausschnitte der RGB gefilterten Aufnahmen.
    Von Oben nach unten für die Doppelspalte 1 bis 4 und für das Beugungsgitter.
}{fig:yng_cutouts}
Um die eindimensionalen Interferenzmuster zu erhalten werden die Ausschnitte aus Abb.\ref{fig:yng_cutouts} in Grauwert Bilder umgewandelt, ihre Pixelwerte über alle Reihen summiert und dann auf den Maximalwert normiert.
Um die räumliche Auflösung des Interferenzmusters zu erhalten wird an den Bildern abgezählt wie viele Kästchen in horizontaler Richtung des Referenzgitters am Bild sind, da die Breite dieser Kästchen mit $d_K$ = 5 mm genau bekannt ist.
Die Interferenzmuster der Doppelspalte werden dann nach Gl.\ref{eq:gru:Int-mult}, Gl.\ref{eq:gru:Int} und Gl.\ref{eq:gru:Beug} mit Spaltparametern $a_i$ und $b_i$ gefittet um die Wellenlänge des Lasers zu erhalten.
Das Interferenzmuster des Gitters Wird nach Gl.\ref{eq:ausw:gitter} mit $\lambda$ = 532 nm, Schirmabstand $z$ aus Kap.\ref{sec:durch:yng} und einer geschätzten beleuchteten Spaltanzahl $N$ = 7 gefittet um den Spaltabstand $b$ zu erhalten.
\begin{equation}
    I_{G,th}(x) = \frac{sin^2\left(N \frac{\pi b x}{\lambda z}\right)}{N^2 sin^2\left(\frac{\pi b x}{\lambda z}\right)}
    \label{eq:ausw:gitter}
\end{equation}
Die erhaltenen Interferenzmuster und Fitfunktionen sind in Abb.\ref{fig:yng_fits} zu sehen.
Die erhaltenen Fitparameter sind in Tab.\ref{tab:yng-Werte} gelistet
\monofig{width=0.7\textwidth}{yng_fits.pdf}{
    Interferenzmuster und deren Fitfunktionen.
    Von Oben nach unten für die Doppelspalte 1 bis 4 und für das Beugungsgitter.
}{fig:yng_fits}
\begin{table}[H]
    \centering
    \caption{
        Ermittelte Fitparameter der Interferenzmuster
    }
    \begin{tabular}{cc} \hline
        Bezeichnung & Wert \\ \hline
        $\lambda_{S1}$  & (552 $\pm$ 13) nm \\ \hline
        $\lambda_{S2}$  & (536 $\pm$ 6 ) nm\\ \hline
        $\lambda_{S3}$  & (528 $\pm$ 3 ) nm\\ \hline
        $\lambda_{S4}$  & (524 $\pm$ 1 ) nm\\ \hline
        $b$  & (125.73 $\pm$ 0.03) $\mu$m \\ \hline
    \end{tabular}
    \label{tab:yng-Werte}
\end{table}

\subsection{Gesetz von Malus}
Die bestimmten Winkel $\beta_1$ und $\beta_2$ bei der maximalen Lichtintensität entsprechen effiktiv einen Winkelversatz der Polarisationsebenen von 0°.
Der Differenzwinkel $\alpha$ ergibt sich zu:
\begin{equation}
    \alpha_i = |\beta_{1,i}- \beta_{1,0}|
\end{equation}
Trägt man nun die Intensität gegen den Winkel $\alpha$ auf kommt das Gesetz von Malus zu Vorschein. Dabei ist sichtbar das einen trigometrischen Verlauf hat.
\monofig{width=0.8\textwidth}{Polar_errobars.png}{Gegenüberstellung der aufgenommenen Datenpunkte zu theoretischen Kurve}{fig:malus}
In Abbildung \ref{fig:malus} ist sichtbar , dass es eine relativ große Diskrepanz zwischen Theorie und Messwert gibt. 
Dies ist vermutlich auf die Polarisation des Lasers zurückzuführen.
Bei der Durchführung des Experiments wurde fälschlicherweise angenommen, dass die Wahl des zu drehenden Polarisators keinen Unterschied macht und nur der Differenzwinkel wichtig ist.
Wird unpolarisiertes Licht als Quelle benutzt, ist dies auch der Fall, doch im Falle des Lasers wird polarisiertes Licht emittiert.
Der Laser ist vermutlich leicht elliptisch polarisiert und hat eine Richtung,in dem das elektrische Feld eine höhere Amplitude hat.
Verdreht man nun den ersten Polarisator, kommt es zu einer Reduktion der Intensität durch die sich verändernde Amplitude des E-Felds und zusätzlich durch den Differenzwinkel zum zweiten Polarisator.
Das bedeutet, dass die Intensität $I_0$ auch von $\alpha$ abhängt.
Da die Polarisatoren früher so eingestellt worden sind, dass die Lichtintensität maximal ist, kann angenommen werden, dass hier die maximale Auslenkung des E-felds liegt.
Die elliptische Auslenkung des E-Feldes kann mittels einer Kurve modeliert werden $\vec{E} \propto  (a \cdot \cos(\alpha), b \cdot \sin(\alpha) )$.
Der Quadrat des Abstandes der Kurve zum Mittelpunkt sollte proportional zur Intensität sein ($I_0(\alpha) \propto |\vec{E}|^2$).
\begin{equation}
    I_0(\alpha) = I_0 \cdot ((a\cos(\alpha))^2 + (b\sin(\alpha))^2)
\end{equation}
Wendet man nun die Korrektur auf die Ausgangsgleichung \ref{eq:malus} an, erhält man:
\begin{equation}
    I(\alpha) = I_0 \cdot ((a\cos(\alpha))^2 + (b\sin(\alpha))^2)\cdot \cos^2(\alpha)
\end{equation}
Fittet man nun die Parameter $a$ und $b$ kann man die Polarisierung ausgleichen.
Zusätzlich wurde aus dem Datenblatt des Lasers erhoben,dass die PER (Polarization extinction ratio ) 5 dB ist.
Nun kann mittels Gl. \ref{eq:per}  (siehe \cite{PER}) das Verhätniss der Leistung der elektromagnetischen Welle in den beiden Halbachsen der Ellipse gefunden werden.
\begin{equation}
    \frac{a}{b} = \frac{P_{max}}{P_{min}} = 10^{\frac{\text{PER}}{10}} = 3.162
    \label{eq:per}
\end{equation}

Es wurde nun auch die Korrektur mittels den bekannten Polarisationskonstanten  des Lasers durchgeführt.
Dabei wird $a =1$ und $b = 1/3.162$ angenommen, da wir den ersten Polaristator auf die große Halbachse der Polarisationellipse des Lasers ausgerichtet haben.
Das modifizierte Gesetz von Malus lautet in diesem Fall :
\begin{equation}
    I(\alpha) = I_0  \biggl( \cos^2(\alpha) + \biggl( \frac{\sin(\alpha)}{3.162} \biggr)^2 \biggr) \cdot \cos^2(\alpha)
\end{equation}
\monofig{width=0.8\textwidth}{Polar_korrektur3.png}{Korrektur der ursprünglichen Gleichung durch Fit und den erhobenen Parameter des Lasers}{fig:korr}
Die Koeffizienten a und b ergeben sich zu: 
\begin{align}
    a &= 0.97 \\
    b &= 0.47 \\
    \frac{a}{b} &= 2.1
    \label{eq:ab}
\end{align}

\subsection{Michelson-Interferometer}
Mit den Messdaten aus Tab.\ref{tab:inter_konz} werden nach Gl.\ref{eq:ausw:hebel} die wirklichen Längenänderungen des Interferometerarms $\Delta x'$ berechnet, da am Spiegel C ein Hebel angebaut ist.
\begin{equation}
    \Delta x' = \frac{\Delta x}{5.3}
    \label{eq:ausw:hebel} 
\end{equation}
Weiters wird dann über Gl.\ref{eq:ausw:lamb} die Wellenlänge des Lasers bestimmt.
\begin{equation}
    \lambda = \frac{2\Delta x'}{N}
    \label{eq:ausw:lamb} 
\end{equation}
Die berechneten Wellenlängen mit Unsicherheit sind in Tab.\ref{tab:ausw:michelson} gelistet
\begin{table}[H]
    \centering
    \caption{
        Ermittelte Wellenlängen durch Michelson Interferometer.
    }
    \begin{tabular}{cc} \hline
        Bezeichnung & Wert \\ \hline
        $\lambda_{1}$  & (600 $\pm$ 50) nm \\ \hline
        $\lambda_{2}$  & (528 $\pm$ 30 ) nm\\ \hline
    \end{tabular}
    \label{tab:ausw:michelson}
\end{table}
%************DISKUSSION***************
\chapter{Discussion}
\label{sec:Discussion}



\printbibliography
\end{document}