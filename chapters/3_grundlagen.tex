%********VORAUSSETZUNGEN & GRUNDLAGEN*********
\chapter{Fundumentals}
\label{sec:Fundumentals}
\section{STM-Imaging}
The Scanning Tunneling Microscope was introduced in 1981 by Gerd Binning and Heinrich Roherer. 
With this measuring technique it is possible to resolve a conductive surface with a precission beyond that of conventional light based Microscopes.
In contrast to other electron based microscopy like Scanning Electron Microscopes (SEM) it uses the quantum mechanical phenomenon of tunneling.
In classical mechanics, objects cannot overcome a potential if their energy $E < V_0$, as observed in gravitational interactions.
This phenomenon is observed for quantum mechanical particles like electrons, which can surpass a potential barrier despite the initial expectation that they should not be able to.
The STM uses this effect by precisely positioning a sharp conductive tip close to the surface and applying a bias voltage.
Most STM are operated in Ultra-High-Vacuum (UHV), where the distance between the tip and the surface represents the tunneling barrier.
By varying the bias voltage, the tunneling probability can be changed, thereby affecting the tunneling current.
If the bias voltage, also referred to as the potential difference, is kept constant, the tunneling current is primarily dependent on the distance between tip and surface.
The tip is moved in the x-,y-plane where a grid is established. 
There are two modes of operation, the constant-height and the constant-current mode.
The latter is especially useful for irregular surfaces, because the tip is moved up and down to keep the tunneling current constant.
The movement signal of the piezos is then converted into height.
In constant-height mode the position of the tip stayes fixed and the tunneling current $I_t$ is measured and converted into height information. 

\newpage