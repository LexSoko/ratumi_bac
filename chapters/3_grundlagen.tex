%********VORAUSSETZUNGEN & GRUNDLAGEN*********
\section{Voraussetzung \& Grundlagen}
\label{sec:voraussetzungen-grundlagen}
\subsection*{Vorwort}
Die in den Grundlagen verwendeten Informationen sowie Gleichungen wurden, falls nicht anders referenziert, aus der Vorbereitungs-PDF \cite{LAB} entnommen.

\subsection{Young'scher Doppelspalt, Beugungsgitter}
Beim Young'schen Doppelspalt (eines der wohl bekanntesten Experimente der Physik) wird eine kohärente Lichtquelle auf einen Doppelspalt geleuchtet (siehe Abb.\ref{fig:gru-doppel-skizz}) um nach dem Schirm ein Interferenzmuster zu erhalten (siehe Abb.\ref{fig:gru-doppel-must}).
Dieses Interferenzmuster $I(x)$ setzt sich nach Gl.\ref{eq:gru:Int-mult} multiplikativ aus dem Interferenzmuster zweier infinitesimal dünner Spalten $I_{I}(x)$ und dem Beugungsmuster eines räumlich ausgedehnten Einzelspalts $I_{B}(x)$ zusammen.
Das Interferenzmuster $I_{I}(x)$ wird in erster Näherung durch Gl.\ref{eq:gru:Int} und das Beugungsmuster $I_{B}(x)$ durch Gl.\ref{eq:gru:Beug} beschrieben.
\begin{equation}
    I(x) = I_{I}(x) \cdot I_{B}(x)
    \label{eq:gru:Int-mult}
\end{equation}
\begin{multicols}{2}
    \begin{equation}
        I_{I}(x) = I_0 \left(1 + cos\left(\frac{2\pi x a}{\lambda z}\right)\right)
        \label{eq:gru:Int}
    \end{equation}\break
    \begin{equation}
        I_{B}(x) = \frac{sin^2\left(\pi x b / \lambda z \right)}{\left(\pi x b / \lambda z \right)^2}
        \label{eq:gru:Beug}
    \end{equation}
\end{multicols}
In den Gleichungen bezeichnet $x$ den Abstand zur Optischen Achse am Schirm, $a$ den Spaltabstant, $\lambda$ die Wellenlänge der Lichtquelle, $z$ die Entfernung zwischen Doppelspalt und Schirm und $b$ die breite der Spalten.
Die drei Gleichungen sind aus \cite{LAB} entnommen worden.
\polyfig{Double-slit_schematic.png}{
    Schematische Skizze eines Doppelspalts mit Abmessungen für genannte Gleichungen. Entnommen aus \cite{Doppelspalt}.
}{fig:gru-doppel-must}{Slit_double_skizz.png}{
    Theoretisches Interferenzmuster am Schirm in rot und in schwarz theoretisches Beugungsmuster eines Einzelspalts. Entnommen aus \cite{Doppelspalt}.
}{fig:gru-doppel-skizz}

\subsection{Polarisations-Analyse}
Im Falle linear polarisierten Lichts lässt sich die transmittierte Intensität durch einen Polarisator mit der Durchlassrichtung entlang der durch den Winkel Null definierten Richtung mithilfe des Gesetzes von Malus berechnen:
\begin{equation}
    I(\alpha) = I_0 \cos^2(\alpha)
    \label{eq:malus}
\end{equation}
Hierbei steht $I(\alpha)$ für die transmittierte Intensität, $I_0$ für die Intensität des einfallenden Lichts und $\alpha$ für den Winkel zwischen der Durchlassrichtung des Polarisators und der Polarisationsebene des einfallenden Lichts.
Die nicht transmittierte Intensität wird je nach Art des Polarisators entweder absorbiert oder reflektiert.
Die Polarisation des Lichts ist von großer Bedeutung für dessen Interferenzfähigkeit. 
Dabei gelten die vier Gesetze nach Fresnel und Arago. 
Wenn Lichtstrahlen in dieselbe Richtung linear polarisiert sind, interferieren sie wie nicht polarisiertes Licht. 
Senkrecht zueinander polarisierte Lichtstrahlen interferieren nicht, es sei denn, sie waren ursprünglich in derselben Polarisationsebene und werden wieder in diese zurückgeführt. 
Wenn senkrecht zueinander polarisierte Lichtstrahlen in dieselbe Polarisationsebene zurückgeführt werden, die sie ursprünglich nicht besaßen, interferieren sie nicht. 
Diese Gesetze lassen sich beispielsweise anhand des Michelson-Interferometers untersuchen.

\subsection{Michelson Interferometer}
Ein Michelson-Interferometer ist ein optisches Instrument, das verwendet wird, um Interferenzmuster von Lichtwellen zu erzeugen. 
Der Strahlengang des Michelson-Interferometers besteht aus einer Laserquelle, einem Strahlteiler, zwei Spiegeln und einem Schirm. 
Der Strahlteiler teilt den Lichtstrahl in zwei Teilstrahlen auf, die durch die beiden Interferometerarme mit unterschiedlichen Längen reflektiert werden. 
Die beiden Teilstrahlen werden am Strahlteiler wieder vereint und interferieren am Schirm, wobei das Interferenzmuster von der Phasendifferenz zwischen den beiden Lichtstrahlen abhängt. 
Die Phasendifferenz ist proportional zum Unterschied der Weglängen $\Delta s$, den die beiden Teilstrahlen zurücklegen.
\begin{equation}
    \Delta \phi = \frac{2 \pi}{\lambda} \Delta s
\end{equation}
Das Interferenzmuster am Schirm wird durch kleine Änderungen in der Richtung des einfallenden Laserstrahls und in der Ausrichtung der Spiegel beeinflusst.
Verwendet man einen Laser als Lichtquelle ist am Schirm aber nur ein kleiner Teil des Beugungsmusteres zu sehen und damit auch die kleinen Änderungen schwer zu erfassen.
Um einen größeren Bereich des Musters abzubilden, wird der Strahl durch eine Linse aufgeweitet.
Dies führt entweder zu einem konzentrischen Interferenzmuster (Linse vor dem Michelson Interferometer) oder zu parallelen Interferenzstreifen (Linse nach dem Michelson Interferometer), je nachdem wo die Linse positioniert ist. 
Die Interferometerarme können auf die gleiche Länge justiert werden, indem man divergierende Strahlen auf virtuelle Lichtquellen zurückführt und die Größe des zentralen Interferenzspots maximiert. 
Eine breitbandige Lichtquelle mit geringer Kohärenzlänge kann auch verwendet werden, um die Interferometerarme einzustellen, wobei die Kohärenzlänge durch Bandpassfilter erhöht werden kann.
Es ist interessant zu beobachten, dass das Interferometer zwei "Ausgänge" hat und dass bei destruktiver Interferenz im Detektorarm die Energie der Lichtwellen nicht verloren geht, sondern am Ausgang zum Laser konstruktive Interferenz auftritt.

\subsection{Unsicherheitsanalyse}
\label{sec:unsichi}
Im gesamten Protokoll, sofern nicht anders angegeben, wird die Größtunsicherheitsmethode laut Gl.\ref{equ:groestunsicherheit} \cite{MMETH} für alle Fehlerrechnungen benutzt.
Hierzu wird das totale Differential der ausgehenden Gleichung gebildet und die Absolutbeträge der Summanden mit der ermittelten Unsicherheit multipliziert.
Alle statistischen Auswertungen werden mit einer statistischen Unsicherheit laut der Studentschen t-Verteilung behaftet.
\begin{equation}
    \varDelta f = \biggl| \frac{\partial f}{\partial x_{1}} \cdot \varDelta x_{1} \biggl| + \biggl| \frac{\partial f}{\partial x_{2}} \cdot \varDelta x_{2} \biggl| + .... + \biggl| \frac{\partial f}{\partial x_{n}} \cdot \varDelta x_{n} \biggl|
    \label{equ:groestunsicherheit}
\end{equation}
\newpage