%************DISKUSSION***************
\section{Diskussion \& Zusammenfassung}
\label{sec:diskussion}
Allgemein sind die Experimente gut verlaufen.
Es wurden die Welleneigenschaften des Lichts untersucht, sowie dessen Polarisationseigenschaften.\\

Bei den Versuchen am Young'schen Doppelspalt und am Beugungsgitter ist gut zu sehen dass die Auswertung stark an der Bildqualität gelitten hat.
Die Aufnahmen sind sehr stark verzerrt und auch mit einer Perspektiven-Verzerrungs-Korrektur sind nicht alle Verzerrungen verschwunden.
Die Kameraoptik erzeugt anscheinend noch sphärische Verzerrungen.
Zusätzlich sind die Aufnahmen stark überbelichtet wodurch die Intensitätsverteilungen nurmehr in ihrer Periodizität zu den theoretischen Verteilungen passten.\\

Bei Untersuchung der polarisationseigenschaften sind ebenfalls dinge aufgefallen.
Wie es scheint ist der Laser elliptisch polarisiert, was allein nur durch den Fit gezeigt wurde.
Die Annäherung durch die Polarisationsparameter des Lasers funktioniert erstaunlich gut und wird womöglich den Effekt am besten beschreiben.
Es ist zwar eine Abweichung nach 90° Drehung zu sehen, jedoch ist der Datenpunkt bei 180° nicht wirklich plausibel, da hier im Prinzip die gleiche Intensität wie bei 0° sein sollte.
Allgemein kann die Abweichung durch die Unsicherheit im Winkel erklärt werden.
Dies macht natürlich die Durchführung des Experiments falsch, jedoch konnte durch Korrektur gezeigt werden, dass das Gesetz von Malus trotzdem gilt .
Zugleich konnte auch die elliptische Polarisation des Lasers nachgewiesen werden.\\

Die Untersuchungen am Michelson Interferometer gelangen an sich recht gut.
Was aber nicht gelang, war die Aufnahme eines komplementären Interferenzmusters.
Es kann dabei aber kein genauer Grund genannt werden wieso dies nicht gelang, aber es wurde eine Hyppothese aufgestellt.
für ein komplementäres Muster müssten die Lichtstrahlen aus den zwei Armen die Richtung Laser zurückgeleitet werden noch mit dem Laser selbst Interferieren.
Dieses Interferenzmuster kann aber mit gegebenem Aufbau nicht beobachtet werden.