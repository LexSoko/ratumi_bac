%************VERSUCHSDURCHFÜHRUNG & MESSERGEBNISSE***********
\section{Versuchsdurchführung \& Messergebnisse}
\label{sec:versuchdurchfuehrung-messergebnisse}

\subsection{Young'scher Doppelspalt, Beugungsgitter}
\label{sec:durch:yng}
Der Aufbau des Experiments wurde gemäß Abb.\ref{fig:double_slit_aufb} realisiert.
Beim Aufbau wurde, zur Einrichtung der Spiegel, mittels einer gerasterten Platte, die Parallelität des Laserstrahls zur Montageplatte und die Umlenkwinkel von 90° eingestellt.
Das Interferenzmuster wurde auf ein kariertes Blatt Papier auf einer Wand im Abstand von $z$ = (252.6 $\pm$ 0.5) cm projiziert.
Die Muster wurden dann mit einer Smartphonekamera, auf einem Holzbrett als Bezugspunkt für konsistente Pixelauflösung, fotografiert.
Da die Kamera aber nicht parallel zur optischen Ebene ausgerichtet werden konnte (sonst wäre der Lichtstrahl bedeckt gewesen) sind die Bilder mit einer Perspektivenverzerrung behaftet.
Die Aufnahmen der Interferenzmuster für die Spalte 1-4 und das Beugungsgitter sind in Abb.\ref{fig:interf_muster} zu sehen.
\begin{figure}[h]
    \centering
    \begin{subfigure}[t]{0.4\textwidth}
        \includegraphics[width=\textwidth]{double_slit_muster1.jpeg}
        \caption{}
        \label{fig:double_slit_muster1}
    \end{subfigure}%
    \begin{subfigure}[t]{0.4\textwidth}
        \includegraphics[width=\textwidth]{double_slit_muster2.jpeg}
        \caption{}
        \label{fig:double_slit_muster2}
    \end{subfigure}
    \begin{subfigure}[t]{0.4\textwidth}
        \includegraphics[width=\textwidth]{double_slit_muster3.jpeg}
        \caption{}
        \label{fig:double_slit_muster3}
    \end{subfigure}%
    \begin{subfigure}[t]{0.4\textwidth}
        \includegraphics[width=\textwidth]{double_slit_muster4.jpeg}
        \caption{}
        \label{fig:double_slit_muster4}
    \end{subfigure}
    \begin{subfigure}[t]{0.4\textwidth}
        \includegraphics[width=\textwidth]{gitter_muster.jpeg}
        \caption{}
        \label{fig:gitter_muster}
    \end{subfigure}

    \caption{
        Aufnahmen der Interferenzmuster der 4 Doppelspalte und des Gitters.
        \ref{fig:double_slit_muster1}: Doppelspalt 1; 
        \ref{fig:double_slit_muster2}: Doppelspalt 2; 
        \ref{fig:double_slit_muster3}: Doppelspalt 3; 
        \ref{fig:double_slit_muster4}: Doppelspalt 4; 
        \ref{fig:gitter_muster}: Gitter.
    }
    \label{fig:interf_muster}
\end{figure}


\subsection{Polarisations-Analyse}

Der Aufbau des Experiments wurde gemäß Abb. \ref{fig:pol_aufbau} vollführt. 
Dabei wurde ein Spiegel um 45° zum Lot des Lichtstrahls ausgerichtet, um diesen um 90° umzulenken.
Es wurde,mittels einer gerasterten Platte, auf die Parallelität des Strahl überprüft.
Hierbei ist entscheidend, dass der gesamte Strahlengang, sowohl horizontal (parallel zu den Befestigungslöcher) als auch vertikal (parallel zum Befestigungstisch) ausgerichtet wird.
Es wurde darauf geachtet, dass der Strahl möglichst Zentral den Intensitätssensor trifft.
Anschließend wurde ein Rohr vor das Luxmeter gestellt, um den Sensor von jeglicher Hintergrundstrahlung abzuschirmen.
Es wurde angenommen, dass der Laser linear polarisiert ist, was die Bestimmung der Vorzugsrichtung der Polarisation erfordert.
Dies wurde getan um die Grundintensität nach dem ersten Polarisator maximal zu halten.
Der Winkel des Polarisators wurde bei dieser Position aufgenommen ($\beta_{1,0}$ = (13 $\pm$ 2)°)
Natürlich gibt es hierbei trotzdem eine Reduzierung der ausgehenden Lichtintensität des Lasers, ausgelöst durch die Interaktion der elektromagnetischen Welle mit den Kristallgitter des Polarisators sowie der Unschärfe in der Polarisationsrichtung des Lasers.
Zuletzt wurde der zweite Polarisator, nach dem ersten, in den Strahlengang gebracht und wieder die Position der maximalen Lichtintensität gesucht ($\beta_{2,0}$= (275 $\pm$ 2)°).
Dieser wurde anschließend in 10° Schritten gedreht und dabei die Lichtintensität aufgezeichnet.


\begin{table}[H]
    \label{tab:pol}
    \centering
    \captionof{table}{
        Experimentell bestimmte Werte. Die Unsicherheit im Winkel wurde mit 2° angenommen und die Unsicherheit der Intensität laut dem Datenblatt \cite{LUX}  berechnet.
        $\beta_1$: Winkel des ersten Polarisators
        $\beta_2$: Winkel des zweiten Polarisators,
        $I$: gemessene Lichtintensität}
    \begin{tabular}{ccc} \hline
    $\beta_1$ / ° & $\beta_2$ / ° & $I$ / Lux \\\hline \hline
    $13$&$275$&$597$\\ \hline
    $20$&$275$&$577$\\ \hline
    $30$&$275$&$503$\\ \hline
    $40$&$275$&$392$\\ \hline
    $50$&$275$&$266$\\ \hline
    $60$&$275$&$164$\\ \hline
    $70$&$275$&$83$\\ \hline
    $80$&$275$&$31$\\ \hline
    $90$&$275$&$9$\\ \hline
    $100$&$275$&$0$\\ \hline
    $110$&$275$& $1$\\ \hline
    $120$&$275$&$13$\\ \hline
    $130$&$275$&$44$\\ \hline
    $140$&$275$&$102$\\ \hline
    $150$&$275$&$192$\\ \hline
    $160$&$275$&$298$\\ \hline
    $170$&$275$&$412$\\ \hline
    $180$&$275$&$504$\\ \hline
    $190$&$275$&$550$\\ \hline
    $195$&$275$&$554$\\ \hline
    \end{tabular}
\end{table}
Im letzten Teil der Polarisations-Analyse wurde der Einfluss eines weiteren Polarisators, welcher zwischen den beiden ursprünglichen Polarisatoren gebracht,untersucht.
Zunächst werden die Filter so eingestellt, dass ihre Polarisationsebenen 90° zu einander stehen (siehe Abb \ref{fig:polmin}).
\duofigcom{pol_2polMin2.jpg}{fig:polmin1}{pol_3polMin1.jpg}{fig:polmin2}{(a) Lichtintensität bei 90° Differenzwinkel. (b) Hier ist die Polarisationsebene des ersten Polarisators zu sehen}{fig:polmin}
Wenn man nun den Filter um 45° dreht, ist ein Intensitätsmaximum zu sehen, da das Licht nun nicht mehr einen Differenzwinkel von 90° zum zweiten Polarisator hat.
\duofigcom{pol_3polMax2.jpg}{fig:polmin1}{pol_3polMax1.jpg}{fig:polmin2}{(a) \& (b) Intensitätsmaximum bei 45 ° Differnezwinkel zu beiden Polarisatoren }{fig:polmin}

\subsection{Michelson-Interferometer}
Um das Michelson Interferometer in Betrieb zu nehmen, muss der Strahlengang eingestellt werden.
Dazu wurde ,wie zuletzt, dieser sowohl parallel zum Tisch als auch parallel zu den Fixierungslöchern eingestellt.
Anschließend wurden die beiden Spiegel (siehe Abb. \ref{fig:michelson_inter}) so eingestellt, dass die zwei reflektierten Strahlen sich am Abbildungsschirm überlagern.
\subsubsection{konzentrisches Interferenzmuster}
Bringt man nun die Linse (siehe Abb \ref{fig:michelson_inter} ,f= 40 mm) vor den Eingang des Interferometers bildet sich ein konzentrisches Interferenzmuster am Abbildungsschirm (siehe Abb. \ref{fig:inter_ges}).
Dabei ist bemerkbar, dass das Zentrum der Interferenzringe mit der Winkeleinstellung des Spiegels $D$ (siehe Abb. \ref{fig:michelson_inter}) verschoben werden kann. 
Hingegen führt die Änderung des Abstandes des Spiegels $C$ ein Verschieben der Interferenzmaxima bzw. -Minima in der Abbildungsebene.
In Abb.\ref{fig:inter_ges} sind zwei konzentrische Interferenzmuster mit unterschiedlichen Einstellungen der Spiegel $D$ und $C$ zu sehen.
\duofigcom{konz_inter_3.jpeg}{fig:inter_klein}{konz_inter_gross.jpeg}{fig:inter_gross}{(a) konzentrisches Interferenzmuster bei größerer Differenz der Armlängen (b) konzentrisches Interferenzmuster bei annäherend gleichen Armlängen}{fig:inter_ges}

Zur Verifikation des komplementären Musters an beiden Ausgängen des Interferometers wurde ein Loch in ein Blatt Papier gestanzt und dieses dann zwischen Linse und Interferometer genau am Brennpunkt der Linse gehalten.
Die beiden Muster sind in Abb.\ref{fig:konz_inter_2} zu sehen.

\begin{figure}[H]
    \centering
    \begin{subfigure}[t]{0.45\textwidth}
        \includegraphics[width=\textwidth]{konz_inter_2.jpeg}
        \caption{}
        \label{fig:konz_inter_2_schirm}
    \end{subfigure}
    \begin{subfigure}[t]{0.45\textwidth}
        \includegraphics[width=\textwidth]{konz_inter_2_komp.jpeg}
        \caption{}
        \label{fig:konz_inter_2_komp}
    \end{subfigure}

    \caption{
        Aufnahmen der Interferenzmuster am Schirm und im Laserarm.
        \ref{fig:konz_inter_2_schirm}: Muster am Schirm; 
        \ref{fig:konz_inter_2_komp}: komplementäres Muster im Laserarm.
    }
    \label{fig:konz_inter_2}
\end{figure}



Um die Wellenlänge des Lasers zu bestimmen wurde das Interferenzzentrum aus der Bildebene gebracht und die Position des Spiegels $C$ variert.
Dabei wurden 2 Messdurchgänge gemacht in denen die Anzahl der Interferenzmaxima-Übergänge $N$ abgezählt und die relative Längenänderung $\Delta x$ aufgezeichnet wurde.
Danach wurde der Spiegel $C$, zur Bestimmung dessen Position bei absolut gleicher Armlängen, in die Position gebracht in der das zentrale Interferenzmaxima am weitesten ausgedehnt wurde (siehe Vgl. Abb.\ref{fig:fig:inter_ges}), und die Position an der Mikrometerschraube $x_0$ abgelesen.
Die Messergebnisse sind in Tab.\ref{tab:inter_konz} aufgelistet.
\begin{table}[H]
    \centering
    \captionof{table}{
        Messungen für die Wellenlängenbestimmung und der Position absolut gleich langer Interferometerarme.
        \textbf{Nr.}: Messnummer; 
        \textbf{Bez.}: Messbezeichnung; 
        $N$: Anzahl der Interferenzmaxima-Übergänge mit Unsicherheit aus grober Abschätzung wie oft verzählt wurde; 
        $\Delta x$: relative Längenänderung an der Mikrometerschraube des Spiegels C mit Unsicherheit der Mikrometerschraube; 
        $x_0$: Position der Mikrometerschraube bei absolut gleichen Weglängen mit abgeschätzter Unsicherheit resultierend daraus wie genau die maximale Ausdehnung des zentralen Interferenzmaxima erkannt werden konnte.
    }
    \begin{tabular}{|c|cc|cc|} \hline
        Nr. & Bez.          & Messwert          & Bez.                  & Messwert \\\hline \hline
        1   & $N$ / 1       & 22 $\pm$ 1        & $\Delta x$ / $\mu$m   & 35 $\pm$ 1 \\ \hline
        2   & $N$ / 1       & 50 $\pm$ 2        & $\Delta x$ / $\mu$m   & 70 $\pm$ 1 \\ \hline \hline
        3   & $x_0$ / mm    & 12.5 $\pm$ 0.1    &                       &  \\ \hline
    \end{tabular}
    \label{tab:inter_konz}
\end{table}


\subsubsection{paralleles Interfernezmuster}
Als nächstes wurde die Linse zwischen Ausgang und dem Abbildungsschirm platziert (Abb. \ref{fig:parallel_danach}). 
Es wurde leider fälschlicherweise die gleiche Linse benutzt, obwohl eine Linse mit einer Brennweite von 16 mm benutzt werden sollte.
Das einbringen der Linse nach dem Ausgang führt nun zu einem parallelen Streifenmuster (Abb. \ref{fig:parallel_muster}).
\duofigcom{michl_paralell_muster.jpg}{fig:parallel_muster}{michl_paralell_linse_danach.jpg}{fig:parallel_danach}{(a) Aufgenommenes paralleles Interferenzmuster, (b) Position der Linse bei der Aufnahme}{fig:parallel_ges}
\subsubsection{Polarisation}
Zuletzt wurde ein Polarisator vor den Eingang des Interferometers gebracht. 
Die Durchlassrichtung des Polarisators wurde mit einem weiteren Polarisator bestimmt, indem dieser solange gedreht wurde bis kein Licht mehr transmittiert wurde.
Dies entspricht einem Winkelversatz von 90° und kann genutzt werden um den Polarisator um 45 grad zum Tisch auszurichten.
Nun wurden die beiden Folienpolarisatoren in die Arme eingebracht und verschiedene Kombinationen der relativen Drehung ausprobiert.
\begin{figure}[h]
    \centering
    \begin{subfigure}[t]{0.4\textwidth}
        \includegraphics[width=\textwidth]{michelson_pol_00_aufb.jpeg}
        \caption{}
        \label{fig:michelson_pol_00_aufb}
    \end{subfigure}%
    \begin{subfigure}[t]{0.4\textwidth}
        \includegraphics[width=\textwidth]{michelson_pol_00_schirm.jpeg}
        \caption{}
        \label{fig:michelson_pol_00_schirm}
    \end{subfigure}
    \begin{subfigure}[t]{0.4\textwidth}
        \includegraphics[width=\textwidth]{michelson_pol_90_aufb.jpeg}
        \caption{}
        \label{fig:michelson_pol_90_aufb}
    \end{subfigure}%
    \begin{subfigure}[t]{0.4\textwidth}
        \includegraphics[width=\textwidth]{michelson_pol_90_schirm.jpeg}
        \caption{}
        \label{fig:michelson_pol_90_schirm}
    \end{subfigure}

    \caption{
        Aufnahmen der Interferenzmuster am Michelson Interferometer bei Polarisierten Teilstrahlen.
        Aufbau \ref{fig:michelson_pol_00_aufb} und Muster am Schirm \ref{fig:michelson_pol_00_schirm} bei zwei horizontal ausgerichteten Polarisatoren; 
        Aufbau \ref{fig:michelson_pol_90_aufb} und Muster am Schirm \ref{fig:michelson_pol_90_schirm} bei einem horizontal  und einem vertikal ausgerichteten Polarisator.
    }
    \label{fig:michelson_pol}
\end{figure}


\newpage