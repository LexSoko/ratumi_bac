\section{Aufgabenstellung}
\label{sec:Aufgabenstellung}
\textit{Die Aufgabenstellung wurde aus \cite{LAB} entnommen.}
\begin{enumerate}
    \item Young'scher Doppelspalt, Beugungsgitter
    \begin{enumerate}
        \item Aufnehmen und vermessen der Beugungsmuster von vier Doppelspalten mit (bekannten) unterschiedlichen Spaltbreiten und Spaltabständen. Berechnen Sie aus den Messwerten die Wellenlänge des Lasers.
        \item Erklären der Details der beobachteten Beugungsmuster durch Vergleich mit den berechneten Mustern.
        \item Bestimmung des Beugungsmuster eines Liniengitters und Vergleich mit berechneten Werten. Berechnung der Gitterkonstante aus Messwerten.
    \end{enumerate}
    \item Wellenfront-Analyse. (Nicht möglich weil Shearing Interferometer in Reparatur)
    \item Polarisation
    \begin{enumerate}
        \item Verifikation des Gesetzes von Malus.
        \item Untersuchung des Einflusses des Durchlasswinkels eines dritten Polarisators zwischen zwei gekreuzten Polarisatoren.
    \end{enumerate}
    \item Michelson Interferometer
    \begin{enumerate}
        \item Justierung eines Interferometers und Erzeugung eines konzentrisches Interferenzmuster. Bestimmung der Wellenlänge des Lasers durch Weglängenänderung. Wiederholung der Messung für ein paralleles Interferenzmuster.
        \item Untersuchung des absoluten Weglängenunterschieds in den beiden Interferometerarmen, sowie Auflösung und Stabilität des Interferometers.
        \item Untersuchung der Rolle der Polarisation auf die Interferenzfähigkeit des Laserlichts
    \end{enumerate}
\end{enumerate}
\newpage