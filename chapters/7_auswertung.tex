%************AUSWERTUNG****************
\section{Auswertung}
\label{sec:auswertung}
Die Auswertung wird mit der Skriptingsprache \verb|Python| \cite{PYTHON} erstellt.
In der gesamten Auswertung werden die Python Libraries, \verb|numpy| \cite{harris2020array}, \verb|pandas| \cite{reback2020pandas}, \verb|scipy| \cite{2020SciPy-NMeth} und \verb|matplotlib| \cite{Hunter:2007}, für statistische Rechenoperationen, das Einlesen und Verarbeiten von Daten sowie das Erstellen von Plots verwendet.
Genaue Funktionsnamen, die zur Berechnung von Messgrößen oder Fit-Parametern verwendet werden, sind in der weiteren Auswertung angegeben.
Des weiteren wird die Library \verb|uncertainties| \cite{UN} für jegliche Unsicherheitsfortpflanzungen (wenn nicht anders angegeben) benutzt, da diese auf dem Prinzip der Größtunsicherheitsmethode Gl. \ref{equ:groestunsicherheit} basiert.


\subsection{Young'scher Doppelspalt, Beugungsgitter}
Da die Aufnahmen aus Abb.\ref{fig:interf_muster} der Beugungsmuster eine Perspektivenverzerrung aufweisen wird diese vorerst mit der Software FIJI (Plugin "Transform -> Interactive Perspective") korrigiert.
Dabei orientiert man sich an einem Äquidistanten Gitter, welches über das Bild gelegt wird und korrigiert durch verzerren von 4 Punkten am Bild.
Die korrigierten Bilder mit vergleichsgitter sind in Abb.\ref{fig:interf_muster_corr_grid} zu sehen.
\begin{figure}[h]
    \centering
    \begin{subfigure}[t]{0.4\textwidth}
        \includegraphics[width=\textwidth]{double_slit_muster1_corr_grid.jpg}
        \caption{}
        \label{fig:double_slit_muster1_corr_grid}
    \end{subfigure}%
    \begin{subfigure}[t]{0.4\textwidth}
        \includegraphics[width=\textwidth]{double_slit_muster2_corr_grid.jpg}
        \caption{}
        \label{fig:double_slit_muster2_corr_grid}
    \end{subfigure}
    \begin{subfigure}[t]{0.4\textwidth}
        \includegraphics[width=\textwidth]{double_slit_muster3_corr_grid.jpg}
        \caption{}
        \label{fig:double_slit_muster3_corr_grid}
    \end{subfigure}%
    \begin{subfigure}[t]{0.4\textwidth}
        \includegraphics[width=\textwidth]{double_slit_muster4_corr_grid.jpg}
        \caption{}
        \label{fig:double_slit_muster4_corr_grid}
    \end{subfigure}
    \begin{subfigure}[t]{0.4\textwidth}
        \includegraphics[width=\textwidth]{gitter_muster_corr_grid.jpg}
        \caption{}
        \label{fig:gitter_muster_corr_grid}
    \end{subfigure}

    \caption{
        Perspektivenkorrigierte Aufnahmen der Interferenzmuster mit eingezeichnetem Vergleichsgitter.
        \ref{fig:double_slit_muster1_corr_grid}: Doppelspalt 1; 
        \ref{fig:double_slit_muster2_corr_grid}: Doppelspalt 2; 
        \ref{fig:double_slit_muster3_corr_grid}: Doppelspalt 3; 
        \ref{fig:double_slit_muster4_corr_grid}: Doppelspalt 4; 
        \ref{fig:gitter_muster_corr_grid}: Gitter.
    }
    \label{fig:interf_muster_corr_grid}
\end{figure}
An den korrigierten Aufnahmen werden dann mit der PIL(Python Image Library) die Pixel in ihren RGB CHannel Werten gefilter, so dass nur Pixel mit Wertigkeit größer als 160 überbleiben.
Dies ist nötig um ein gutes Interferenzsignal zu erhalten.
Danach wird aus jedem Bild nur der Bereich ausgeschnitten in dem das Interferenzmuster zu sehen ist.
Die Ausgeschnittenen Bereiche sind in Abb.\ref{fig:yng_cutouts} zu sehen.
\monofig{width=0.7\textwidth}{yng_cutouts.pdf}{
    Ausschnitte der RGB gefilterten Aufnahmen.
    Von Oben nach unten für die Doppelspalte 1 bis 4 und für das Beugungsgitter.
}{fig:yng_cutouts}
Um die eindimensionalen Interferenzmuster zu erhalten werden die Ausschnitte aus Abb.\ref{fig:yng_cutouts} in Grauwert Bilder umgewandelt, ihre Pixelwerte über alle Reihen summiert und dann auf den Maximalwert normiert.
Um die räumliche Auflösung des Interferenzmusters zu erhalten wird an den Bildern abgezählt wie viele Kästchen in horizontaler Richtung des Referenzgitters am Bild sind, da die Breite dieser Kästchen mit $d_K$ = 5 mm genau bekannt ist.
Die Interferenzmuster der Doppelspalte werden dann nach Gl.\ref{eq:gru:Int-mult}, Gl.\ref{eq:gru:Int} und Gl.\ref{eq:gru:Beug} mit Spaltparametern $a_i$ und $b_i$ gefittet um die Wellenlänge des Lasers zu erhalten.
Das Interferenzmuster des Gitters Wird nach Gl.\ref{eq:ausw:gitter} mit $\lambda$ = 532 nm, Schirmabstand $z$ aus Kap.\ref{sec:durch:yng} und einer geschätzten beleuchteten Spaltanzahl $N$ = 7 gefittet um den Spaltabstand $b$ zu erhalten.
\begin{equation}
    I_{G,th}(x) = \frac{sin^2\left(N \frac{\pi b x}{\lambda z}\right)}{N^2 sin^2\left(\frac{\pi b x}{\lambda z}\right)}
    \label{eq:ausw:gitter}
\end{equation}
Die erhaltenen Interferenzmuster und Fitfunktionen sind in Abb.\ref{fig:yng_fits} zu sehen.
Die erhaltenen Fitparameter sind in Tab.\ref{tab:yng-Werte} gelistet
\monofig{width=0.7\textwidth}{yng_fits.pdf}{
    Interferenzmuster und deren Fitfunktionen.
    Von Oben nach unten für die Doppelspalte 1 bis 4 und für das Beugungsgitter.
}{fig:yng_fits}
\begin{table}[H]
    \centering
    \caption{
        Ermittelte Fitparameter der Interferenzmuster
    }
    \begin{tabular}{cc} \hline
        Bezeichnung & Wert \\ \hline
        $\lambda_{S1}$  & (552 $\pm$ 13) nm \\ \hline
        $\lambda_{S2}$  & (536 $\pm$ 6 ) nm\\ \hline
        $\lambda_{S3}$  & (528 $\pm$ 3 ) nm\\ \hline
        $\lambda_{S4}$  & (524 $\pm$ 1 ) nm\\ \hline
        $b$  & (125.73 $\pm$ 0.03) $\mu$m \\ \hline
    \end{tabular}
    \label{tab:yng-Werte}
\end{table}

\subsection{Gesetz von Malus}
Die bestimmten Winkel $\beta_1$ und $\beta_2$ bei der maximalen Lichtintensität entsprechen effiktiv einen Winkelversatz der Polarisationsebenen von 0°.
Der Differenzwinkel $\alpha$ ergibt sich zu:
\begin{equation}
    \alpha_i = |\beta_{1,i}- \beta_{1,0}|
\end{equation}
Trägt man nun die Intensität gegen den Winkel $\alpha$ auf kommt das Gesetz von Malus zu Vorschein. Dabei ist sichtbar das einen trigometrischen Verlauf hat.
\monofig{width=0.8\textwidth}{Polar_errobars.png}{Gegenüberstellung der aufgenommenen Datenpunkte zu theoretischen Kurve}{fig:malus}
In Abbildung \ref{fig:malus} ist sichtbar , dass es eine relativ große Diskrepanz zwischen Theorie und Messwert gibt. 
Dies ist vermutlich auf die Polarisation des Lasers zurückzuführen.
Bei der Durchführung des Experiments wurde fälschlicherweise angenommen, dass die Wahl des zu drehenden Polarisators keinen Unterschied macht und nur der Differenzwinkel wichtig ist.
Wird unpolarisiertes Licht als Quelle benutzt, ist dies auch der Fall, doch im Falle des Lasers wird polarisiertes Licht emittiert.
Der Laser ist vermutlich leicht elliptisch polarisiert und hat eine Richtung,in dem das elektrische Feld eine höhere Amplitude hat.
Verdreht man nun den ersten Polarisator, kommt es zu einer Reduktion der Intensität durch die sich verändernde Amplitude des E-Felds und zusätzlich durch den Differenzwinkel zum zweiten Polarisator.
Das bedeutet, dass die Intensität $I_0$ auch von $\alpha$ abhängt.
Da die Polarisatoren früher so eingestellt worden sind, dass die Lichtintensität maximal ist, kann angenommen werden, dass hier die maximale Auslenkung des E-felds liegt.
Die elliptische Auslenkung des E-Feldes kann mittels einer Kurve modeliert werden $\vec{E} \propto  (a \cdot \cos(\alpha), b \cdot \sin(\alpha) )$.
Der Quadrat des Abstandes der Kurve zum Mittelpunkt sollte proportional zur Intensität sein ($I_0(\alpha) \propto |\vec{E}|^2$).
\begin{equation}
    I_0(\alpha) = I_0 \cdot ((a\cos(\alpha))^2 + (b\sin(\alpha))^2)
\end{equation}
Wendet man nun die Korrektur auf die Ausgangsgleichung \ref{eq:malus} an, erhält man:
\begin{equation}
    I(\alpha) = I_0 \cdot ((a\cos(\alpha))^2 + (b\sin(\alpha))^2)\cdot \cos^2(\alpha)
\end{equation}
Fittet man nun die Parameter $a$ und $b$ kann man die Polarisierung ausgleichen.
Zusätzlich wurde aus dem Datenblatt des Lasers erhoben,dass die PER (Polarization extinction ratio ) 5 dB ist.
Nun kann mittels Gl. \ref{eq:per}  (siehe \cite{PER}) das Verhätniss der Leistung der elektromagnetischen Welle in den beiden Halbachsen der Ellipse gefunden werden.
\begin{equation}
    \frac{a}{b} = \frac{P_{max}}{P_{min}} = 10^{\frac{\text{PER}}{10}} = 3.162
    \label{eq:per}
\end{equation}

Es wurde nun auch die Korrektur mittels den bekannten Polarisationskonstanten  des Lasers durchgeführt.
Dabei wird $a =1$ und $b = 1/3.162$ angenommen, da wir den ersten Polaristator auf die große Halbachse der Polarisationellipse des Lasers ausgerichtet haben.
Das modifizierte Gesetz von Malus lautet in diesem Fall :
\begin{equation}
    I(\alpha) = I_0  \biggl( \cos^2(\alpha) + \biggl( \frac{\sin(\alpha)}{3.162} \biggr)^2 \biggr) \cdot \cos^2(\alpha)
\end{equation}
\monofig{width=0.8\textwidth}{Polar_korrektur3.png}{Korrektur der ursprünglichen Gleichung durch Fit und den erhobenen Parameter des Lasers}{fig:korr}
Die Koeffizienten a und b ergeben sich zu: 
\begin{align}
    a &= 0.97 \\
    b &= 0.47 \\
    \frac{a}{b} &= 2.1
    \label{eq:ab}
\end{align}

\subsection{Michelson-Interferometer}
Mit den Messdaten aus Tab.\ref{tab:inter_konz} werden nach Gl.\ref{eq:ausw:hebel} die wirklichen Längenänderungen des Interferometerarms $\Delta x'$ berechnet, da am Spiegel C ein Hebel angebaut ist.
\begin{equation}
    \Delta x' = \frac{\Delta x}{5.3}
    \label{eq:ausw:hebel} 
\end{equation}
Weiters wird dann über Gl.\ref{eq:ausw:lamb} die Wellenlänge des Lasers bestimmt.
\begin{equation}
    \lambda = \frac{2\Delta x'}{N}
    \label{eq:ausw:lamb} 
\end{equation}
Die berechneten Wellenlängen mit Unsicherheit sind in Tab.\ref{tab:ausw:michelson} gelistet
\begin{table}[H]
    \centering
    \caption{
        Ermittelte Wellenlängen durch Michelson Interferometer.
    }
    \begin{tabular}{cc} \hline
        Bezeichnung & Wert \\ \hline
        $\lambda_{1}$  & (600 $\pm$ 50) nm \\ \hline
        $\lambda_{2}$  & (528 $\pm$ 30 ) nm\\ \hline
    \end{tabular}
    \label{tab:ausw:michelson}
\end{table}