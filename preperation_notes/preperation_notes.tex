%Authoren: Max Mustermann
%Vorlage: Max Mustermann


%
\documentclass[headheight=30pt]{scrartcl}

\usepackage[utf8]{inputenc} % utf8x durch utf8 ersetzt wegen biblatex
\usepackage[T1]{fontenc}
\usepackage{amsmath,amssymb,amstext, graphicx}
\usepackage[ngerman]{babel}
\usepackage{csquotes}
\usepackage{pdfpages} %zum importieren des Deckblattes
\usepackage[headsepline]{scrlayer-scrpage}
\usepackage{lastpage}
\usepackage[style=numeric]{biblatex} %Literaturverwaltung
\usepackage{tabularx} %für die Legende im Gruundlagen Oszi Bild
\usepackage{float}
\usepackage{physics}

\addbibresource{literatur.bib} %Literatur-Resourcen
\pagestyle{scrheadings}

\date{\today{}, Graz}
\author{maxjost42}
\title{Vorbereitung}

\clearpairofpagestyles
\ihead{\today{} \\ }
\chead{Authorname \\ Thema}
\ohead{Kurs: Labor \\ }
\cfoot{\pagemark \, / \, \pageref{LastPage}}


\begin{document}


\newpage
\paragraph{1. In welchem Verhältnis stehen die Gewichte der transportierten Massen verschiedener Substanzen bei gleichem Ladungsdurchfluß?} ~
\begin{equation}
    F = \frac{ItM}{mz} \rightarrow m = \frac{Fz}{ItM}
    \label{equ:FaradayConst}
\end{equation}
Startend mit Gl. \ref{equ:FaradayConst} aus \cite{LAB} wird das Massenverhältnis $m_1/m_2$ her:
\begin{equation}
    \frac{m_1}{m_2} = \frac{\frac{Fz_1}{ItM_1}}{\frac{Fz_2}{ItM_2}} = \frac{\frac{z_1}{M_1}}{\frac{z_2}{M_2}} = \frac{z_1M_2}{z_2M_1}
    \label{equ:MassVerh}
\end{equation}
Aus Gl. \ref{equ:MassVerh} ist nun das massenverhältnis zu sehen, welches nur von der $z$-Wertigkeit und der molaren Masse der beiden Substanzen abhängig ist.


\paragraph{2. Das Atomgewicht von Silber mit der Ordnunszahl 47 ist 107.868. Begrunden Sie die Unganzzahligkeit!} ~ \\
Das Atomgewicht von Silber wird in der atomaren Masseneinheit \textit{u} \cite{ATOMM} angegeben und durch die Anzahl an Protonen und Neutronen im Atomkern berechnet.
Protonen und Neutronen besitzen aber nicht genau ein gewicht von 1 u sondern $m_{p^+}$ = 1,007276 u \cite{PROTM} und $m_n$ = 1,008664 u \cite{NEUTM}.
Summiert man diese gewichter auf mit der Protonenanzahl $n_{p^+}$ = 47 und Neutronenanzahl $n_{n}$ = 107 - 47 = 60 so ergibt sich die atomare Masse von $^{107}Ag$ mit $m_{^{107}Ag}$ = 107.868 u.


\paragraph{3. Der vorliegende Elektrolyt wurde durch Auflösen von AgNO$_3$-Kristallen in Wasser hergestellt.
    Beim Lösen der Kristalle zerfällt das Kristallgitter, die Ionen werden frei (Dissoziation), und können sich unter Einfluß eines äußeren elektrischen Feldes zwischen den Molekülen des Lösungsmittels bewegen.
    Begrunden Sie die Dissoziation der Ionen!
    Hinweis: Die elektrostatische Anziehung zwischen den Ionen hängt nach dem Coulomb’schen Gesetz von der Dielektrizitätskonstante ab} ~ \\
Sieht man sich Gl. \ref{equ:elFeld} (das Coulombsche Gesetz) für die Kraft $\vb{F_1}$ die auf die Ladung $q_1$ wirkt an so stellt man fest,
dass bei Materialien mit größerer relativer Permittivität $\varepsilon_r$ diese Kraft kleiner sein muss.
\begin{equation}
    \vb{F_1} = \frac{q_1q_2}{4\pi\varepsilon_r \varepsilon_0 |\vb{r_{12}}|^3} \vb{r_{12}}
    \label{equ:elFeld}
\end{equation}
Silbernitratkristalle besitzt eine relative Permittivität von $\varepsilon_{rAgNO_3}$ = 9 \cite{PERMS} und Wasser eine relative Permittivität von $\varepsilon_{rH_2O}$ = 1.77 \cite{PERMW}.
Somit kann sich Wasser mit größerer Kraft um die Ag und NO$_3$ Ionen anordnen als besagte Ionen untereinander.
\newpage


\printbibliography
\end{document}