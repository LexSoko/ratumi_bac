%************AUSWERTUNG****************
\chapter{Results}
\section{Pentacenequinone}
Unfortunately inspite of three attempts to get a stable evaporation process it was not possible to get sub-monolayer/monolayer growth.
This lead to difficulties regarding image resulution,but it was still possible to observe makrostructures forming.
Pentacenequinone seems to form round island that are randomly scattered and have no clear periodic structure.
If looking closely at the edges of the terasses, it is visible that the sharp edges that are typical for Ag(100) are rough.
This could indicate a full monolayer present.
\duofigcom{C:\\Bac_Arbeit\\ratumi_bac\\bac_template\\graphics\\ready_images_pentacene\\41nm_pentacene_3dig.png}{fig:penta_3dig}{C:\\Bac_Arbeit\\ratumi_bac\\bac_template\\graphics\\ready_images_pentacene\\41nm_pentacene_1dig.png}{fig:penta_1dig}{seas}{}

\newpage
\section{2-Hydrogen Phthalocyanin on Ag(100)}
\label{sec:results}The STM and LEED Images of 2-Hydrogen Phthalocyanin (2HPc) on Ag (100) show the Formation of an ordered Structure.
As seen in the Image \ref{fig:2hpcAg_o} the 2HPc  is most likely in $\sqrt{17} \times  \sqrt{17}\text{R}14$ Phase with an relative rotation of the molecular axis (MA) in respect to the [010] direction of 7 ° (see figure \ref{fig_fig:2hpcAg_o}). 
Furthermore the STM Images reveal that the distance between the nearest molecules is 
15 A and that the unit cell of the monolayer is also square.
\monofig{width = 0.7\textwidth}{C:\\Bac_Arbeit\\ratumi_bac\\bac_template\\graphics\\ready_images_phthalo\\2Hpc_on_Ag100_3.PNG}{STM image of 2HPc/Ag(100) taken in constant-current mode (monolayer regime).(a) Constant-height Image of Ag(100) as reference  . U = 53.5 mV , I = 1 nA. MA: Molecular axis.}{fig:2hpcAg_o}  
\noindent The Molecule itself appears as a 4-fold symmetric cross with a dark center in the middle or as 2-fold symmteric cross with two opposing isoindole units seeming brighter than the other two for positive polarity . 
It oriented so that the benezene ring of one 2HPc molucule is pointed towards the benzene ring of the nearest neighbor. 
For negative biasvoltages (filled state imaging), near the fermi level a two fold symmetric orbital is seen. 
\duofigcom{C:\\Bac_Arbeit\\ratumi_bac\\bac_template\\graphics\\ready_images_phthalo\\2HPc_on_Ag100_LUMO_1.png}{fig:LUMO_1}{C:\\Bac_Arbeit\\ratumi_bac\\bac_template\\graphics\\ready_images_phthalo\\2Hpc_on_Ag100_HOMO.PNG}{fig:HOMO_1}{(\ref{fig:LUMO_1}) Constant-current of 2Hpc/Ag(100). Two distinct LUMO orientations can be seen and their DFT-Simulation (xxx Puschnig ref) counterparts. U = -50.0 mV , I = 0.05 nA. (\ref{fig:HOMO_1}) Same spot imaged in constant-current mode at positive Bias-Voltages. U = 750 mV , 0.02 nA.  }{}
\noindent This is most likely due to charge transfer from the metal substrate , leading to an occupied LUMO and thus a charged molecule.  
There are two degenerated LUMOs (same Energy), with the probability density concentrated around two opposing isoindole units. 
This makes the Molecule reduce its symmetry from 4-fold to 2-fold.
One difference of the two degenerated LUMOs is the orientation of the probability density, as it is rotated 90° in respect to eachother.
The STM images (see Fig.\ref{fig:LUMO_1}) reveal a connection beween two opposing isoindole units, which is in agreement  with the simulated orbitals using density functional theory (DFT).  
There is no favored orientation seen in the STM images, it is seemingly random which suggest that the occupation probability of the two LUMOs is the same. 
Its worth mentioning that the LUMO is also visible for positive bias which suggest two things: the negativly charged molecules LUMO is singly or fractionaly occupied and therefore it is possible for a electron to tunnel into that state or it tunnels into the other not occupied LUMO. 
The latter is suported by the 

\section{Magnesium-Oxide (MgO)}
Before depositing 2Hpc (2H-phthalocyanine) onto the MgO crystal structure, the quality of the surface was assessed using scanning tunneling microscopy (STM), followed by capturing a low-energy electron diffraction (LEED) image.
The MgO forms sharp rectangular Islands with the most edges along the [011] and [01$\bar{1}$] directions as expected.
Due to a similar lattice constant and an equivalent fcc crystal lattice it can be assumed that there are local coherent phase boundarys.
It is shown in the figures \ref{fig:21nm_mgo} \& \ref{fig:10nm_mgo} that most of the islands were 15 - 20 nm wide.
The Images show a adsorbtion pattern consistent with literature. 
\duofigcom{C:\\Bac_Arbeit\\ratumi_bac\\bac_template\\graphics\\ready_images_MgO\\21nm_MgO.png}{fig:21nm_mgo}{C:\\Bac_Arbeit\\ratumi_bac\\bac_template\\graphics\\ready_images_MgO\\10nm_MgO.png}{fig:10nm_mgo}{Both (a) and (b) taken at 3.5 V Biasvoltage, most likely sub-monolayer to monolayer regime}{}

\newpage
\section{2-Hydrogen Phthalocyanine on Mg0}
The STM images of 2Hpc on MgO(100)/Ag(100) show again the formation of ordered structure that persists across the the whole crystal lattice.
Evident is the presence of linear defects, on which sites the 2Hpc Phase is shifted or even disconnected. 
This is most likely due to the fact that the underlying MgO long-range order is desturbed by the 3.2 \% mismatch of the Ag an MgO lattice.
Therefore resulting in incoherent sites, which give rise to intermediate phases with different orientation.
In figure \ref{fig:2hpcMgO} the phase of the molecular adsorbant was estimated.
This was done using the Ag(100) lattice vectors, as it almost identical to the MgO lattice, where Oxygen is adsorbed on top of the Ag atoms.
The unit cell of 2HPc on MgO is tilted in direction of the [01$\bar{1}$] plane and is also larger than on Ag which can be attributed to the the larger lattice constant.
The phase was estimated to $\sqrt{22.5} \times  \sqrt{22.5}\text{R}(-18.4)$ in respect to the Ag(100) lattice, where the angle measurement is in respect to the [010] plane.
Note that this was the most common phase and the large scale distribution of different molecular species is not known.
In contrast to 2HPc on Ag the molecular is almost mirrored in respect to the [001] plane with an molecular axis angle of -6°.
\monofig{width = 0.7\textwidth}{C:\\Bac_Arbeit\\ratumi_bac\\bac_template\\graphics\\ready_images_phthalo\\2Hpc_on_MgO.PNG}{STM image of 2HPc on Mg0(100)/Ag(100) taken in constant-current mode (monolayer regime).(a) Constant-height Image of Ag(100) as reference  . U = 200 mV , I = 1 nA. MA: Molecular axis.}{fig:2hpcMgO}  
\noindent
For further verifaction of the observed structere a LEED picture at 15 eV was taken after analysis with the STM.
The LEED clearly supports the mainly observed phase, but looked diffused which can come from additional phases or additional diffraction caused by the molecular structure.
Additionally a 2D-FFT was performed on the the image in figure \ref{fig:2hpcMgO} which alignes nicely with the observed diffraction.
\duofigcom{C:\\Bac_Arbeit\\ratumi_bac\\bac_template\\graphics\\ready_images_phthalo\\2HPc_on_MgO_FFT.PNG}{fig:2hpcMgoFFT}{C:\\Bac_Arbeit\\ratumi_bac\\bac_template\\graphics\\ready_images_phthalo\\2HPc_on_MgO_LEED.PNG}{fig:2hpcMgoLEED}{(a) 2D-FFT of image in \ref{fig:2hpcMgO}, (b) LEED picture taken at $E_{kin}$ = 15 eV}{fig:fftnLEED}

\duofigcom{C:\\Bac_Arbeit\\ratumi_bac\\bac_template\\graphics\\ready_images_phthalo\\2HPc_on_MgO_neg_400mV_rings.PNG}{fig:onMgo400mV}{C:\\Bac_Arbeit\\ratumi_bac\\bac_template\\graphics\\ready_images_phthalo\\2HPc_on_MgO_700mV_rings.PNG}{fig:onMgO700mV}{(a)}{fig:onMgO}
