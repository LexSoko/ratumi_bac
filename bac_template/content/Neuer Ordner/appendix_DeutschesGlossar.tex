\chapter{Glossar Englisch-Deutsch
\label{sec:glossar}}

Zur Besseren Übersicht ist hier ein Glossar der deutschsprachigen Übersetzungen der Fachbegriffe aus der Literatur gegeben. 
Ebenso sind die hierbei verwendeten und in der Literatur gebräuchlichen Abkürzungen der Fachbegriffe in der Literatur angeführt.

\vspace{5mm}

{\centering
% \begin{landscape}
\begin{table}[h!]
\caption{Glossar der häufig verwendeten Begriffe auf Englisch mit ihrer deutschen Übersetzung und ihrer verwendeten Abkürzung.}
\label{tab:glossar}
\tabcolsep=10pt
\hfill

\begin{tabular}{lll}

\hline\hline
\multicolumn{1}{c}{Abkürzung} & 
\multicolumn{1}{c}{Englisch} & 
\multicolumn{1}{c}{Deutsch} \\



\hline
EOS & equation of state & Zustandsgleichung \\
FDU & first dredge up & maximales Durchmischungsereignis \\
HRD & Hertzsprung-Russell diagram & Hertzsprung-Russell-Diagramm \\
MS & main sequence & Hauptreihe \\
PMS & pre-main sequence & vor der Hauptreihe \\
RGB & red-giant branch & Roterriesenast\\
TRGB & tip of the red-giant branch & Spitze des Rotenriesenastes\\
\hline
\end{tabular} \hfill~
\end{table}
% \end{landscape}

\par}
