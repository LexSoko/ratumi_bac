% !TEX root = ../thesis-example.tex
%
\pdfbookmark[0]{Acknowledgement}{Acknowledgement}
\addcontentsline{toc}{chapter}{Acknowledgements}
\chapter*{Acknowledgement}
\label{sec:acknowledgement}
\vspace*{-10mm}

% (1) First lets thank in-house personal for their continuous support that way too often goes unnoted. Take away the part you didn't use, e.g. if your thesis does not contain work at the Lustbühel observatory or computations on the HPC cluster, delete these sentences. 
This thesis made use of the infrastructure, resources and observational facilities of the Department for Geophysics, Astrophysics and Meteorology (IGAM) at the Institute of Physics of the University of Graz. 
I thank DI\,Roland Maderbacher and Mag.\,Klaus Huber for maintaining the IT-infrastructure, software installations, the server farm and providing technical support, whenever needed. The team of UniIT is thanked for maintaining and supporting the use of the high performance cluster (HPC), used for computations presented in this thesis.
I thank Dr.\, Rainer Kuschnig, Mag.\,Robert Greimel, Josef Ramsauer, and Dr.\, Martin Leitzinger for their technical support and expertise at the Lustbühel Observatorium Graz (OLG) operated by the University of Graz. 
This thesis made use of infrastructure, which was supported by NAWI Graz. 


% (2) now list the missions, telescopes, data bases or code packages you have used. Copy here the predefined sentence you are requested on the project webpage to cite. Also make sure that you cite the requested papers, as mentioned in the text below.
%take what you need for the mission you use:
I thank the people behind the space missions, whose data we were using in this thesis. This thesis includes data collected by the \textit{Kepler} and the \textit{TESS} missions. Funding for the \textit{Kepler} mission is provided by the NASA Science Mission directorate. Funding for the \textit{TESS} mission is provided by the NASA Explorer Program. 
%GAIA 
This work also has made use of data from the European Space Agency (ESA) mission \textit{Gaia}, processed by the \textit{Gaia} Data Processing and Analysis Consortium (DPAC). Funding for the DPAC has been provided by national institutions, in particular, the institutions participating in the \textit{Gaia} Multilateral Agreement.
%BRITE
Based on data collected by the BRITE Constellation satellite mission, designed, built, launched, operated and supported by the Austrian Research Promotion Agency (FFG), the University of Vienna, the Technical University of Graz, the University of Innsbruck, the Canadian Space Agency (CSA), the University of Toronto Institute for Aerospace Studies (UTIAS), the Foundation for Polish Science \& Technology (FNiTP MNiSW), and National Science Centre (NCN).
This work has utilized the MESA stellar evolutionary code package, Modules for Experiments in Stellar Astrophysics
\citep[MESA][]{Paxton2011, Paxton2013, Paxton2015, Paxton2018, Paxton2019}. 
The MESA EOS is a blend of the OPAL \citep{Rogers2002}, SCVH
\citep{Saumon1995}, FreeEOS \citep{Irwin2004}, HELM \citep{Timmes2000},
PC \citep{Potekhin2010}, and Skye \citep{Jermyn2021} EOSes.
Radiative opacities are primarily from OPAL \citep{Iglesias1993,Iglesias1996}, with low-temperature data from \citet{Ferguson2005}
and the high-temperature, Compton-scattering dominated regime by
\citet{Poutanen2017}.  Electron conduction opacities are from
\citet{Cassisi2007}. Nuclear reaction rates are from JINA REACLIB \citep{Cyburt2010}, NACRE \citep{Angulo1999} and additional tabulated weak reaction rates \citet{Fuller1985,Oda1994,Langanke2000}.Screening is included via the prescription of \citet{Chugunov2007}. Thermal neutrino loss rates are from \citet{Itoh1996}.
%Lustühel Obersvatory
%
This research has made use of the SIMBAD database, operated at CDS, Strasbourg, France.
%
This research has made use of the Exoplanet Follow-up Observation Program website, which is operated by the California Institute of Technology, under contract with the National Aeronautics and Space Administration under the Exoplanet Exploration Program. 
%
\par
\textit{Software:} \texttt{Python} \citep{10.5555/1593511}, 
\texttt{numpy} \citep{numpy,Harris_2020},  
\texttt{matplotlib} \citep{4160265},  
\texttt{scipy} \citep{2020SciPy-NMeth}, 
\texttt{pandas} \citep{reback2020pandas, mckinney-proc-scipy-2010}.
This research made use of \texttt{astropy} \citep{astropy:2013, astropy:2018}., a community-developed core Python package for Astronomy. 

% (3) FUNDING (if it is only sentence, append it to the above paragraph)

%acknowledge sources of funding like a travel grant or a Erasmus fellowship, e.g.:
I gratefully acknowledge the Austrian Science Fund (FWF): P30949-N36 (PI: Temmer) for supporting this project. 
This work was supported with funding of the Dr. Heinrich-Jörg Foundation at the Faculty of Natural Sciences at the Karl-Franzens University of Graz.
The authors acknowledge the support from ERASMUS+ grant number 2017-1-CZ01-KA203-035562.