\chapter{Example for the presentation of programming code \label{chap:appendixCode}}

%In this section the code developed for this work is presented. Mention here the general dependencies you used in your development environment.
%\systemDependencies{}

\section{System dependencies \label{sec:SystemDependencies}}
The presented code is written for Python 3.7. 
\Table{tab:PyModules} details the used standard and specialized python packages and modules and is structured as follows. 
\begin{itemize}
\item The package and module names are listed in the first column. Specialized modules are preceeded with the abbreviation of the parent package, whose abbreviated package name is inidacted in brackets.

\item The version number of the used library is indicate in the second column.

\item A brief module description is provided in the third column. 
%
\end{itemize}
The top panel reports the standard python packages, contained in the standard python distribution. The bottom panel lists specialized packages, which require individual download. 
(Modified table based on Table A.1 from Rafael Goldgruber's Bachelor thesis).


\begin{table}[h!]
  %{\centering
  \vspace{3mm}
  \tabcolsep=8pt
  \caption{Required Python modules for the presented code.}
    \begin{tabular}{p{0.2\textwidth}
                    p{0.1\textwidth}
                    p{0.6\textwidth}}
    \hline\hline
    Package / Module & Version & Modul Description\\
    \hline
    math  & $-$ & Access mathematical functions; \\
    os    & $-$    & Miscellaneous 
    operating system interfaces  \\
    glob & $-$ & Creates iterable lists from folder content \\

    matplotlib &  3.3.1 & Core package for scientific computation \& plotting \\
    numpy & 1.19.1 & Core package for numerical computing \\

    \hline
    astropy (ap) & 4.0.1 & Community Python Library for Astronomy  \\
    \textit{~~ap}.astroquery & 0.4   & Querying astronomical web databases \\
    \textit{~~ap}.vizier & 0.4   & Importing online data and catalogues, published along with papers \\
    \textit{~~ap}.io.pyfits & & Reading and operating with FITS files \\
    mesareader & $-$ & Reading MESA history and profile output files\\
    %datetime & $-$ & supplies classes for manipulating dates and times  \\
    %jdcal &  1.4.1 & functions for converting between Julian dates and calendar dates \\
    %PIL   & $-$   & Python Imaging Library  \\
    %unicodedata & $-$  & access to the Unicode CharacterDatabase\\
    \hline
    \end{tabular}
  \label{tab:PyModules}
  
%\tablefoot{ }  
\end{table}

\section{Program to name new Planets \label{cod:HellowWorld}}

Provide a short description of the following script and give a brief description of the key settings or input format. The example below shows how to insert the code of a whole program from its source file.

\lstinputlisting[label=code:sampleCode, 
frame=Tb,
caption= HelloPlanet.py: Sophisticated code to name new worlds.,
language=python]{code/helloWorld.py}


\vspace{5mm}
\begin{table}[h!]
    \caption{Option menu for helloWorld.py (\Listing{code:sampleCode}), \label{cod:helloWorld}}

    %\centering
    \hfill\begin{tabular}{p{0.15\textwidth}p{0.15\textwidth}p{0.6\textwidth}}
    \hline\hline
    Type     & Argument &  Functionality \\ \hline
    InCode    & if \hbox{xY==True} & Description of what will happen if this option is used. \\
    CmdL    & -n & Description of what will happen if this option is used. \\
    ReqIn    & parameter name & Description of what will happen if this option is used. \\
    \hline
    \end{tabular}\hfill~
    
    \tablefoot{Usage options: 
    \textit{InCode} curial options to be set inside the code before excecution of the program,
    \textit{CmdL} options set in the command line, 
    \textit{ReqIn} requested user input during the execution of the program;
    }
    \label{tab:functionalityTable}
\end{table}

\section{Example for long code listing \label{chap:appendixMesaInlists}}

This section presents styles to present longer code snippets. To save printing space and cost, these are typset in a two-column format. 

The combined presentation of \Listing{code:inlist1} and \Listing{code:inlist2} show how two snippets can be shown. \Listing{code:longListing} depicts how long code snippets or full programs can be included as multi-column and multi-page element.

Also for these listings a preamble stating the system requirements is needed.


\begin{landscape}
\begin{multicols}{2}
\begin{lstlisting}[caption=inlist1 \label{code:inlist1}]
&star_job
      show_log_description_at_start = .true.
      pgstar_flag = .true.
/ ! end of star_job namelist

&controls
      extra_terminal_output_file = 'log1' 
      log_directory = 'LOGS1'
/ ! end of controls namelist

&pgstar
      History_Panels1_win_flag = .true.
      History_Panels1_win_width = 5
      History_Panels1_win_aspect_ratio = 1.0 
      
      History_Panels1_title = 'Orbital evolution'
      History_Panels1_num_panels = 2
      
      History_Panels1_yaxis_name(1) = 'period_days'
      History_Panels1_other_yaxis_name(1) = 
      'lg_mstar_dot_1' 
      History_Panels1_yaxis_name(2) = 'Jdot' 
      History_Panels1_other_yaxis_name(2) = 
      'binary_separation' 
/ ! end of pgstar namelist

\end{lstlisting}
%##########################################################################
\begin{lstlisting}[caption=inlist2 \label{code:inlist2}]
&star_job
      show_log_description_at_start = .false.

/ ! end of star_job namelist

&controls
      extra_terminal_output_file = 'log2' 
      log_directory = 'LOGS2'
/ ! end of controls namelist

&pgstar













/ ! end of pgstar namelist
\end{lstlisting}
\end{multicols}
%\tablefoot{}
% \end{table*}
\end{landscape}


\begin{landscape}

\begin{lstlisting}[language=Fortran,
caption = Example for a multi-column and multi-page code listing. 
\label{code:longListing},frame=T]
\end{lstlisting} 
\begin{lstlisting}[language=Fortran,multicols=2,frame=none]
! inlist to evolve a 15 solar mass star

! For the sake of future readers of this file,
! ONLY include the controls you are actually using. 
! DO NOT include all of the other controls that 
! simply have their default values.

&star_job

  ! begin with a pre-main sequence model
    create_pre_main_sequence_model = .true.

  ! save a model at the end of the run
    save_model_when_terminate = .false.
    save_model_filename = '15M_at_TAMS.mod'

  ! display on-screen plots
    pgstar_flag = .true.

/ !end of star_job namelist


&controls

  ! starting specifications
    initial_mass = 15 ! in Msun units

  ! options for energy conservation 
     use_dedt_form_of_energy_eqn = .true.
     use_gold_tolerances = .true.

  ! stop when the star nears ZAMS (Lnuc/L > 0.99)
    Lnuc_div_L_zams_limit = 0.99d0
    stop_near_zams = .true.

  ! stop when the center mass fraction
  of h1 drops below this limit
    xa_central_lower_limit_species(1) = 'h1'
    xa_central_lower_limit(1) = 1d-3

/ ! end of controls namelist

&pgstar

  ! MESA uses PGPLOT for live plotting and 
  ! gives the user a tremendous amount of control 
  ! of the presentation of the information.

  ! show HR diagram
  ! this plots the history of L,Teff over many 
  ! timesteps
    HR_win_flag = .true.

  ! set static plot bounds
    HR_logT_min = 3.5
    HR_logT_max = 4.6
    HR_logL_min = 2.0
    HR_logL_max = 6.0

  ! set window size (aspect_ratio = height/width)
    HR_win_width = 6
    HR_win_aspect_ratio = 1.0


  ! show temperature/density profile
  ! this plots the internal structure at single 
  ! timestep
    TRho_Profile_win_flag = .true.

  ! add legend explaining colors
    show_TRho_Profile_legend = .true.

  ! display numerical info about the star
    show_TRho_Profile_text_info = .true.

  ! set window size (aspect_ratio = height/width)
    TRho_Profile_win_width = 8
    TRho_Profile_win_aspect_ratio = 0.75

/ ! end of pgstar namelist
\end{lstlisting} 
\vspace{-\baselineskip}
\noindent\rule{\linewidth}{0.4pt}

\end{landscape}
