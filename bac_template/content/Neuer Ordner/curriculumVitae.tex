% --- LaTeX CV Template - S. Venkatraman ---
\pdfbookmark[0]{Curriculum vitae}{Curriculum Vitae}
\addcontentsline{toc}{chapter}{Curriculum Vitae}
\chapter*{Curriculum Vitae\\Mag. Nikolaus Copernicus, PhD}
\chaptermark{}

\label{sec:cv}
%\subsection*{Bibliographic information and overview}
\begin{table}[h!]
    %\hline\hline
    \begin{tabular}{p{0.25\textwidth}p{0.75\textwidth}}
     \physnormal{Born}  & 
     February 19, 1473 \\%[0.5em] 
     \physnormal{Citizenship}  & 
     Poland \\
\end{tabular}\hfill~
    
\end{table}


\section*{Research positions and academic education}
%\physRules

\subsection*{Current academic position}
\begin{table}[h!]
    \begin{tabular}{p{0.25\textwidth}p{0.74\textwidth}}
    \physnormal{Affilitation \& \newline academic address}  & 
    Institute for Physics, Karl Franzens University Graz, \newline Universitätsplatz 5/II, NAWI Graz, 8010 Graz, Austria \\
    
    \smallskip \physnormal{Contact}  & 
    \smallskip  nicolaus.copernicus@uni-graz.at \\

    %\smallskip {Academic repository identifiers}  & 
    %\smallskip  ORCID: xxx, \newline GitHub: xxx \\
    \end{tabular}\hfill~
    
\end{table}

\subsection*{Research Positions}
\begin{longtable}{p{0.25\textwidth}p{0.74\textwidth}}
    {since March 2020}  & 
    Doctoral researcher at the 
	Department for Astronomy, Inst. of Physics, NAWI Graz, University of Graz, Austria \\[0.5em]
    
    {2019, March  \newline $-$ 2020, March }  & 
    Pre-doctoral researcher at the 
	Department for Astronomy, Inst. of Physics, NAWI Graz, University of Graz, Austria \\[0.5em]


\end{longtable}


\subsection*{Academic Education}
\begin{longtable}{p{0.25\textwidth}p{0.74\textwidth}}

    {2017, February \newline $-$ 2019, March}  &	University of Graz, Austria: Master of Natural Sciences in Physics (Graduation: March 32, 2019), Title of Thesis\\[0.5em]
    
    {2014, October \newline $-$ 2017, February}  &	University of Vienna, Austria : Bachelor of Natural Sciences in Astronomy (Graduation: February, 30, 2017)\\[0.5em]

\end{longtable}
\clearpage

\section*{Academic Recognition}
%\physRules

\subsection*{Invited talks at conferences and seminars}
\begin{itemize}
    
    \item \textbf{Title of invited contribution}, title of conference or occasion, length of contribution, (date, hosting institution)
    
    %\item \textbf{Title of invited contribution}, title of conference or occasion, length, (date, hosting institution)
    
    %\item \textbf{Title of invited contribution}, title of conference or occasion, length, (date, hosting institution)
    
\end{itemize}


\subsection*{Grants \& Awards}
\begin{itemize}
    
    \item \textbf{Name of achievement}, awarding entity or agency, (awarded amount, duration)
    
    %\item \textbf{Name of achievement}, awarding agency, (awarded amount, duration)
    
    %\item \textbf{Name of achievement}, awarding agency, (awarded amount, duration)
    
\end{itemize}



\subsection*{Service to the community \& memberships in academic organization}
\begin{itemize}

    \item \textbf{Refereeing activity for scientific journals}, give concise details

    \item \textbf{Membership in relevant professional societies, consortia or collaborations}, give concise details
    
    \item \textbf{Function in associations}, give concise details

\end{itemize}

\subsection*{Extended scientific or exchange stays}
\begin{itemize}

    \item Institution, program / purpose, give concise details

    
\end{itemize}

\newpage
\section*{Teaching Portfolio}%\physRules
List here the courses you have assisted in or taught until now.
\begin{itemize}
    
    \item \textbf{Name of Course}, 
    (type of course: e.g. lecture, exercises, practicum, ...) 
    Semester 202X, 
    ECTS of the course, 
    Bachelor's / Master's Curriculum,
    list your co- or lead-lecturer, 
    description of your responsibilities in this course.

    \item \textbf{Name of Course}, 
    (type of course: e.g. lecture, exercises, practicum, ...) 
    Semester 202X, 
    ECTS of the course, 
    Bachelor's / Master's Curriculum,
    list your co- or lead-lecturer, 
    description of your responsibilities in this course.

\end{itemize}

\physbf{Nota bene:} If you use this template for your Habilitationsschrift, replace this section by the structure and content of the teaching portfolio, as recommended\footnote{\href{https://lehrkompetenz.uni-graz.at/de/service/lehrportfolios/}{https://lehrkompetenz.uni-graz.at/de/service/lehrportfolios/}} by the \physit{Zentrum für Lehrkompetentz (ZLK) of Graz University.}


\section*{Observing Portfolio}%\physRules

The candidate has obtained XXX nights of accumulated observing experience of at international facilities:
\begin{itemize}
    \item Observatory name and location, telescope, technique, number of nights
    %\item Observatory name and location, telescope, technique, number of nights
    %\item Observatory name and location, telescope, technique, number of nights
\end{itemize}



\subsection*{Proposals lead as Principle investigator}
\begin{itemize}
    \item \textbf{Title of successful proposal}, 
    awarded observing time (observing semester),  
    observatory, telescope \& instrument specification
    
    %\item \textbf{Title of successful proposal}, 
    %awarded observing time (observing semester),  
    %observatory, telescope \& instrument specification
    
    %\item \textbf{Title of successful proposal}, 
    %awarded observing time (observing semester),  
    %telescope \& instrument specification, 
    
\end{itemize}


\subsection*{Proposals contributed to as Co-investigator}
\begin{itemize}

    \item \textbf{Title of successful proposal}, 
    awarded observing time (observing semester),  
    name of PI,
    observatory, telescope \& instrument specification
    
    %\item \textbf{Title of successful proposal}, 
    %awarded observing time (observing semester),  
    %name of PI,
    %observatory, telescope \& instrument specification 
    
\end{itemize}
