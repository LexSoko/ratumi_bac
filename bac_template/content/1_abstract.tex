\chapter*{Abstract}
\label{sec:abstract}
Since the development of the Scanning Tunneling Microscope (STM) in 1981 it was used to get an insight into the electronic structure of molecules. \cite{binnig1982surface}
Especially the adsorption of organic molecules on metal or metal-oxide substrates peaked interest in the STM-Imaging field because of their semiconducting properties and their capacity for self-assembly.
Additionally they promise cheap, flexible and tunable alternatives to conventional inorganic semiconducters in electronic and optoelectronic engineering fields \cite{OTERO2017105}.
In this thesis two adsorbates are studied using Scanning Tunneling Microscopy: Pentacenequinone and 2-Hydrogen-Phthalocyanine which fall into this category and are extensivly researched \cite{wang2012structures}\cite{sperl2011controlled}\cite{thomas1990phthalocyanine}\cite{hollerercharge}.
%Both are organic semiconducters with many ap 
The primary objective of this thesis is to comprehend and classify the adsorption characteristics of those two molecules on Ag(100), and for 2HPc also on ultrathin MgO(100) layers grown on Ag(100).

 
