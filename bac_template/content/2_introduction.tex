%***************ANMERKUNGEN*******************
\chapter{Introduction}
\label{sec:anmerkung}
The study of molecular adsorption on solid surfaces is an important and influential field within surface science.
This thesis is a comprehensive exploration of the adsorption behaviour of two organic molecules, Pentacenequinone (PQ) and 2 Hydrogen Phtalocyanine (2HPc), on Ag(100) and MgO(100)/Ag(100).
The present study delves into the phenomenon of Fractional and Integer Charge Transfer (FCT and ICT) and the formation of ordered structures. 
Organic molecules with semiconductive properties such as 2HPc, exhibit significant potential in opto-electronic technologies.
The enhancement of device efficacy and functionality is reliant on their surface interactions. 
Furthermore Phthalocyanines are used as enzyme-like catalysts , liquid crystals, in photovoltaic cells or as photodynamic reagents for cancer therapy. \cite{wang2012structures}
The versatility of these molecules arises from their ability to form coordination complexes with metals, giving them the name Metal-Phthalocyanines (MPc).\cite{wang2012structures}
This process can change their optical or electronic properties, while leaving the inherent geometric structure unchanged \cite{wang2012structures}.
Hence, it is important to understand the intricacies of their adsorption behavior, charge transfer mechanisms, and structural order formation to achieve technological progress. \\
%It was even shown for a similar group of organic molecules the porphyrins, to be precise 2H-tetraphenylporphyrin (2H-TPP), that the self-metallation on thin MgO films can occur.
%This process called metallation is promoted by charge transfer and can be controlled by tuning the work function of the MgO(100)/Ag(100) subtrate \cite{egger2021charge}.
%Therefore something similar for 2HPc is expected.
For 2H-tetraphenylporphyrin (2H-TPP), a molecule similar to 2HPc, self metallation has been observed on MgO(100) thin films. \cite{egger2021charge}
Interestingly, in this case charge transfer seems to play a crucial role in this process.
In 2H-TPP the 4 peripheral phenyl rings are tilted relative to the plane of the molecular macrocycle, leading to a relativly high adsorption height of the macrocycle .
Charge transfer transfer reduces this adsorption height, enabling self metallation (see Fig. \ref{fig:2htpp}).
\monofig{width=0.7\textwidth}{2http.jpg}{Selfmetallation precess of 2H-TPP on MgO(100)/Ag(100) \cite{egger2021charge}}{fig:2htpp}
Naturally, this leads to the question whether those two processes, self-metallation and charge transfer, are also connected in such a way for completely flat molecules, such as 2HPc.
The aim of this thesis is to understand and categorize the adsorption behavior of both
adsorbates on an Ag(100) single crystal and on a MgO(100) thin film grown on
Ag(100), with the main focus on 2-Hydrogen-Phthalocyanine (2HPc).
PQ is exclusively imaged on Ag(100), where the focus is to observe bigger structures without imaging individual molecules.



