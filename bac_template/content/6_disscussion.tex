%************DISKUSSION***************
\chapter{Discussion}
\label{sec:Discussion}

\section{2HPc/Ag(100)}
\textbf{Observation of one Phase:}\\
The molecule is mainly observed in a $\sqrt{34} \times  \sqrt{34}\text{R}31$ phase with a molecular axis rotated (11.6 $\pm$ 1.2)° relative to [010] (vertical direction).
Interestingly the Molecular Axis doesnt line up with the [010] and [001] direction of the underlying silver lattice.
This may indicate sigificant interactions between the nearest molecular neighbors.\\
\textbf{Charge Transfer and Symmetry Reduction}\\
The data indicates the shifting of the LUMO under the Fermi Level.
This consequently leads to charge transfer that is observed as a partially occupied LUMO.
The LUMO is seen well near the Fermi Energy, which indicates Fermi Level pinning.
Furthermore the LUMO is also seen for postive bias which supports that it is partially occupied.
The occurence of partially charged molecules leads to symmetry reduction from $C_4$ (four-fold) to $C_2$ (two-fold).
It is notable that for some molecules the dircetion of the prominent LUMO lobe is switched for positive biases and vice versa.
This may come from the degeneracy of the LUMO which has two flipped Orbitals that have an angle of 90° in respect to each other.
Therefore leading to a splitting of the degenerate energy levels and a more energetically favourable orbital to tunnel in to.
Further investigation is needed to categorize this effect.
  




\section{2HPc/MgO(100)/Ag(100)}
The 2HPc/MgO/Ag system gives rise to suprising phenomena that deviates from the expected observation.\\
\textbf{Observation of multiple Phases:}\\
In contrast to 2HPc on Ag(100) there are atleast three distinct phases, that are seen on the MgO film.
%These phases show that the molecule adsorbtion is more versatile when it comes to adsorbing on an thin dielectric film like MgO.
The observation of multiple phases of 2HPc on MgO indicates a more complex  adsorbtion behaviour on the dielectric film than on Ag(100).
Additionally there are more defects present, which can be explained by the incoherent boundaries between Ag(100) and MgO(100). \\
\textbf{ICT and SOMO/SUMO splitting:} \\ 
Our results show that 2HPc exibits a different electronic structure upon adsorption on MgO(100) than on Ag(100).
This can be explained by the commonly known Integer Charge Transfer trough the thin dielectric layer.
This results in an SOMO/SUMO splitting and could provide valuable insights in similar molecular systems.\\

\textbf{Symmetry Change:}\\ 
An unexpected observation was the deviation from the 2-fold symmetry expected for a charged molecule on MgO(100).
As seen in Fig. \ref{fig:onMgO} an unexpected four-fold symmetry is observed, which could be a hint for a metallation process occuring.
This deviation could not be explained with the collected data and warrants further investigation.

